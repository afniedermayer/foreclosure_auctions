\documentclass[11pt,twopage]{article}
%%%%%%%%%%%%%%%%%%%%%%%%%%%%%%%%%%%%%%%%%%%%%%%%%%%%%%%%%%%%%%%%%%%%%%%%%%%%%%%%%%%%%%%%%%%%%%%%%%%%%%%%%%%%%%%%%%%%%%%%%%%%%%%%%%%%%%%%%%%%%%%%%%%%%%%%%%%%%%%%%%%%%%%%%%%%%%%%%%%%%%%%%%%%%%%%%%%%%%%%%%%%%%%%%%%%%%%%%%%%%%%%%%%%%%%%%%%%%%%%%%%%%%%%%%%%
\usepackage{amssymb}
%\usepackage[round]{natbib}
\usepackage[round]{natbib}
\usepackage{makeidx}
\usepackage{graphicx,amssymb}
\usepackage{caption}
\usepackage{subcaption}
\usepackage{amsmath}
\usepackage{a4,latexsym,amsmath,amstext,graphicx,amssymb}
\usepackage{amsthm}
\usepackage{bbm}
\usepackage{color}

% %
%\usepackage{setspace}

\PassOptionsToPackage{hyphens}{url}\usepackage{hyperref}
%\usepackage{hyperref}


% turn on comments:
\newcommand{\mycomment}[1]{\textbf{[#1]}}
\newcommand{\AN}[1]{\textcolor{red}{[AN: #1]}}
\newcommand{\AS}[1]{\textcolor{blue}{[AS: #1]}}
\newcommand{\PX}[1]{\textcolor{green}{[PX: #1]}}

% turn off comments:
%\newcommand{\mycomment}[1]{}
%\newcommand{\AN}[1]{}
%\newcommand{\AS}[1]{}
%\newcommand{\PX}[1]{}


\setcounter{MaxMatrixCols}{10}


\DeclareMathOperator{\owhom}{OWHOM}
\DeclareMathOperator{\poly}{poly} \DeclareMathOperator{\dlog}{DL}
\DeclareMathOperator{\image}{img} \DeclareMathOperator{\prob}{Pr}
\DeclareMathOperator{\IF}{IF} \DeclareMathOperator{\ELSE}{ELSE}
\DeclareMathOperator{\NP}{NP} \DeclareMathOperator{\DEXP}{DEXP}
\DeclareMathOperator{\DMEXP}{DMEXP}
\DeclareMathOperator{\DPOW}{DPOW}
\DeclareMathOperator{\DMEXPPOW}{DMEXP-POW}
\DeclareMathOperator{\GSROOT}{GSROOT}
\DeclareMathOperator{\desc}{desc}
\newcommand{\ol}{\overline}
\newcommand{\ul}{\underline}
%\newenvironment{hypothesis}[Hypothesis]
\newtheorem{claim}{Claim}
{\bf}{\it}
\newtheorem{ourproblem}{Problem}
{\bf}{\it}
\newtheorem{problem}{Problem}
{\bf}{\it}
\newtheorem{assumption}{Assumption}
{\bf}{\it}
\newtheorem{definition}{Definition}
{\bf}{\it}
\newtheorem{proposition}{Proposition}
{\bf}{\it}
\newtheorem{remark}{Remark}
{\bf}{\it}
\newtheorem{lemma}{Lemma}
{\bf}{\it}
\newtheorem{theorem}{Theorem}
{\bf}{\it}
\newtheorem{corollary}{Corollary}
{\bf}{\it}
\newtheorem{conjecture}{Hypothesis}
{\bf}{\it}
\textwidth=450pt \textheight=610pt \marginparwidth=0pt
\topmargin=0pt \marginparsep=0pt \oddsidemargin=10pt
\evensidemargin=10pt
\newcommand{\draft}[1]{}
\newcommand{\be}{\begin{equation}}
\newcommand{\ee}{\end{equation}}
\newcommand{\bes}{\begin{equation*}}
\newcommand{\ees}{\end{equation*}}
\newcommand{\bea}{\begin{eqnarray}}
\newcommand{\eea}{\end{eqnarray}}
\newcommand{\beas}{\begin{eqnarray*}}
\newcommand{\eeas}{\end{eqnarray*}}

\begin{document}

\title{Foreclosure Auctions} \author{
  \\
  Andras Niedermayer\\
  {University of Mannheim}\\
  \\
  Artyom Shneyerov\\
  {Concordia University, CIREQ, CIRANO}\\ \\
  Pai Xu\\
  {University of Hong Kong, Hong Kong} }

\date{This Draft:  \today}

\thispagestyle{empty}
\maketitle

\begin{abstract}

We develop a novel theory of real estate foreclosure auctions, which
have the special feature that the lender acts as a seller for low and
as a buyer for high prices. The theory yields several empirically testable predictions concerning the strategic behavior of the agents, both under symmetric and asymmetric information. Using novel data from Palm Beach County (FL, US), we find evidence of both strategic bidding and asymmetric information, with the lender being the informed party. Moreover, the data are consistent with mortgage securitization dissipating the original lender's information and leading to less aggressive bidding in the auction.
%
%We get two predictions: (i) lenders'
%bids are bunched at the amount owed, (ii) in the presence of a common value
%component the probability of selling to a third-party bidder is non-monotone
%and discontinuous in the lender's maximum bid. Using novel data from Palm Beach County (FL, US), we show
%that (i) and (ii) are consistent with the
%data. Further, our theory and our data allow us to confirm the wide-held suspicion that lenders
%of securitized mortgages are less informed than lenders of non-securitized mortgages. 
%Our theory also allows a welfare analysis of judicial versus non-judicial foreclosures.
\end{abstract}

\setlength{\baselineskip}{1.5\baselineskip}
%\doublespacing

\noindent \textbf{Keywords: } foreclosure auctions, common value component, securitization \\
\textbf{JEL Codes: } to be added


\section{Introduction}

Foreclosure auctions of real estate have a substantial economic
impact. In 2013, 609,00 residential sales in the U.S. were foreclosure related
(foreclosure auctions or sales of real estate
owned by a lender), amounting to 10.3\% of total residential sales.\footnote{See \url{http://www.realtytrac.com/Content/foreclosure-market-report/december-and-year-end-2013-us-residential-and-foreclosure-sales-report-7967}.}
Foreclosures were one of the major concerns during the economic
crisis, with many home owners losing their property because of the
drop of real estate prices. %[ADD CITATIONS]

We develop a novel theory of foreclosure auctions, 
incorporating the institutional specificities of foreclosures. A
foreclosure auction is run by a government agency (in our data set
the Clerk and Comptroller's Office of Palm Beach County, FL) after a mortgagee stopped making
payments to the lender. The lender, typically a bank, and third-party
bidders, typically real estate brokers, 
participate in this auction. Payments up to the amount owed (the
judgment amount) are paid to the lender, payments above the judgment
value, if any, are paid to the owner of the property. The owner
typically does not participate. In such an auction, the bank
essentially acts as a seller below the judgment amount, and as a buyer above the judgment amount.

An important feature of foreclosure auctions is that the bank is likely the better informed party. There are several reasons for this. First, the bank assesses the value of the property when it grants the mortgage. Second, the value of the property is greatly affected by how much the owner invested in maintenance. The bank typically has more information concerning such maintenance investments: the bank can e.g.\ observe whether the owner was in financial distress for a longer period of time and hence neglected maintenance. Third, a major problem with foreclosed property is that the original owner may abandon the house, take with them everything they can get their hands on, and sometimes even actively damage the property. The bank's previous interactions with the owner may provide it with better information about the likelihood of this behavior. The law, however, prevents other bidders (e.g.\ real estate brokers) to enter and inspect the property. This  further enhances the informational advantage of the bank.\footnote{This legal requirement is meant to protect the borrower, who may still live in the property during the foreclosure auction. An exception is that in some mortgage contracts there are provisions which allow the lender's representative to enter a vacant property to make repairs and to provide regular maintenance such as turning utilities on and off (see \url{http://www.nolo.com/legal-encyclopedia/deceptive-foreclosure-practices-when-banks-treat-occupied-homes-vacant.html}). For such mortgage contract, this gives an additional informational advantage to the lender.}

The broker's expected payoff depend both of the bank's information and on their own idiosyncratic components (signals). The latter may reflect their own value added to the property.\footnote{When visiting websites of several brokers participating in the foreclosure auctions in our dataset, we could see that they often specialize in renovating foreclosed properties.} This value added may be largely independent of the baseline value of the house, so we assume that the broker and bank signals are independent. The relevance of the bank's information to the broker creates a common value component in the auction. However, unlike the standard common value model where buyers have relevant information, here is it assumed that only the bank is the informed party. 

We derive participants' equilibrium bidding strategies in a foreclosure auction with a common
value component, which, as we have argued, should be relevant in foreclosure auctions. However, it is also interesting to consider the symmetric information environment, where the bank's information is known to the brokers. This independent private value (IPV) setup is nested within our general model. Under IPV, brokers bid their own type as in a standard English auction. The bank bids its valuation plus a markup for low valuations, just as a seller would bid. The bank's bid takes into account the classical trade-off of a seller: a higher bid lowers the bank's probability of selling the house, but increases the bank's revenue in case of sale. However, one side of this trade-off disappears for banks that bid exactly the judgment amount ($v_J$): a higher bid still lowers the probability of sale, but does not lead to higher revenues for the bank, since the additional revenue goes to the original owner. This change of the trade-off causes a bunching of banks' bids at the $v_J$. If the bank's valuation is sufficiently high -- higher than $v_J$- the bank does not want to sell the property, but would want to keep it itself. Hence, for high valuations, the bank bids its own type, just as a buyer would do in an English auction.

Under asymmetric information, the bank's bid also serves as a signal about the bank's information about the quality of the house. For values sufficiently below $v_J$, when the bank effectively acts as a seller, the bank's bid involves a markup over its value. But we also show that on top of this markup, it involves a \emph{signalling premium} that makes the bid higher than what it would have been under symmetric information. The intuition for the signalling premium is that, under asymmetric information, the bank should have a lower probability of sale given any bid in order to make the bid incentive compatible. Under asymmetric information, the bank would have an additional incentive to deviate from equilibrium by bidding higher, as such a bid would signal a higher quality to the brokers and will, other things equal, lead to a higher probability of sale, which is beneficial to the bank. To compensate for this, the equilibrium price should be higher under asymmetric information.


The bunching result from the symmetric information setup carries over to the asymmetric information one. However, bunching has a surprising implication in the presence of a common value component: there cannot be an equilibrium where the bank's biding strategy is continuous around the bunch. The reason for this is that brokers who are supposed to drop slightly below $v_J$ will instead prefer to wait for the price to rise above $v_J$, as they would then be certain to obtain a significantly higher house quality at a slightly higher price. Since brokers will not have an incentive to drop out at prices slightly below $v_J$, neither will have the bank: increasing the bid to $v_J$ will lead to a higher revenue for the bank, while not changing the probability of selling. This logic implies that there has to be a discontinuity in the bank's bidding strategy below $v_J$. We show that the bank's equilibrium bidding strategy is strictly increasing and continuous for low valuations, exhibits a discontinuous jump to $v_J$ and bunching at $v_J$, and is equal to the bidding strategy of a buyer for valuations above $v_J$.


Our
theory has two striking empirically testable implications. First,
there is bunching of banks' bids at the judgment amount $v_J$. Second, in the presence of both auctions with symmetric and asymmetric information, one should expect to observe the
following. For values slightly below $v_J$, the probability of the bank selling the house to a third-party bidder first increases with the bank's bid and then drops down discontinuously at $v_J$. This is due to the following selection effect. For prices slightly below $v_J$, the symmetric information auctions are selected into the observable sample. This is because there is a \emph{gap} in the bank's bids below $v_J$ under asymmetric information. For prices at and slightly above $v_J$, both symmetric and asymmetric information auctions are selected. As the probability of sale (``demand") is lower under asymmetric information due to the signalling premium, the observable demand will exhibit a downward jump at $v_J$. Further, for bank's bids just below $v_J$, the probability of sale increases with the bank's bid. The reason is that the closer the bank's bid approaches $v_J$ from below, the more one gets a selection into symmetric information auctions. This selection leads to a higher observed probability of sale, because of the aforementioned signalling premium in the asymmetric information auctions. This \emph{demand discontinuity} and \emph{non-monotonicity} due to the selection effect are testable predictions in the environment where both symmetric and asymmetric information auctions are present.

%The probability of sale (i.e. the probability that a
%realtor wins the auction) decreases with the seller's reserve price
%for prices far below and far above the judgment amount. However, just
%below and just above the judgment amount, the probability of sale
%\emph{increases} with the price, this effect being due to a selection
%effect.

We have collected a novel data set with foreclosure auctions from
Palm Beach County in Florida. We have data on 12,788 auctions from
2010 to 2013 with the
total judgment amount \$4.2bn and the total sale amount (the sum of the winning bids) 
\$0.5bn. The data reveals that bunching indeed occurs
at the judgment amount. We also observe that the probability of sale
increases with the bank's bid just below the judgment amount. Further, demand exhibits a downward discontinuity at the judgment amount. This points, as we have argued, to the presence of auctions with both symmetric and asymmetric information in our dataset.

%As we have argued, the distinguishing feature of the asymmetric information environment is the signalling premium. Controlling for the value of the house, we should expect the bank to bid higher under symmetric information. But at the same time, we have also argued that our sample contains both symmetric and asymmetric information auctions. Nevertheless, we are able to address the signalling premium directly by looking separately at \emph{securitized} mortgages, where, as we shall argue, the bank may not have superior information. By estimating the bidding strategies for securitized and non-securitized mortgages, we can empirically verify the presence of the signalling premium. 

Further empirical support for asymmetric information is provided by considering securitized and non-securitized mortgages. 
%
%As we have argued, asymmetric information leads to a signalling premium in bidding. By comparing securitized and non-securitized mortgages, we are able to show the existence of a signalling premium directly, thus providing further empirical support for the existence of asymmetric information in foreclosure auctions. 
The securitization of mortgages was a major concern during the financial crisis.
Originating banks securitized their mortgages through
securitization agencies (mostly the Government Sponsored Enterprises
Freddie Mac and Fannie Mae), the securitized assets were then sold on
the capital market. There is a wide held suspicion that moral hazard played an important role during the financial crisis: the suspicion is that banks did not exert sufficient effort in gathering information before granting mortgages and hence excessively
granted mortgages and securitized them, shifting
the burden to the holders of securitized assets and Government
Sponsored Entities (and ultimately to the taxpayer). 

Different levels of information gathering for securitized and non-securitized mortgages have empirically testable implications for foreclosure auctions: a bank holding a non-securitized mortgage is expected to be better informed about the value of the property than a bank holding a securitized mortgage. This is for two reasons. First, the bank is more likely to have gathered information about the loan-to-value ratio of a property for non-securitized mortgages. Second, the bank's estimate of the probability of default of a borrower is informative about the value of the property at the time of the foreclosure (see \cite{qi2009loss}). A reason for the relation between the probability of default and the value of the property is that borrowers in financial distress are less likely to invest in the maintenance of their property.\footnote{Causality can, of course, also go in the opposite direction: home owners with a higher loan-to-value ratio have more of an incentive to default. For our purposes, it is immaterial in which direction the causality goes. In either case, superior information about the probability of default also leads to superior information about the value of the property.} % [ADD SOME
% CITATIONS HERE]

%We can divide the foreclosure auctions into those related to a
%securitized mortgage and those related to a non-securitized mortgage. 
One would therefore
expect the bank to have less private information about the quality of the house for securitized
mortgages than for non-securitized mortgages. This implies that the non-monotonicity and the discontinuity of the probability of sale as a function of the bank's bid should be present for non-securitized mortgages, but these both features are less saleint, or not even statistically discernible for securitized mortgages. An analysis of the data reveals that this is indeed the case.

Next, we consider a version of our model where the broker's valuation depends on the bank's information  \emph{linearly}, with coefficient $\alpha$ measuring the degree of bank's informativeness. A higher $\alpha$ corresponds to more precise information concerning the common value component. We show that the bank will bid more aggresively if $\alpha$ is higher. This is \emph{both} because of the direct effect on the broker values, and the indirect effect due to the higher incentive to signal the favourable information to the brokers. 
%
%We provide direct evidence that asymmetric information plays a bigger role for the foreclosure auctions of non-securitized  by showing that the signalling premium is larger. 
In order to be able to estimate the bank's bidding strategy, we collected additional data on the resale prices and the tax assessment values of foreclosed properties. Using the resale price and the assessment value as two independent noisy signals for banks' valuations, we constructed the distribution of banks' valuations for securitized and non-securitized mortgages. Matching the quantiles of banks' valuation distributions with banks' bid distributions gives us the (average) bidding function. Having an estimate of the bidding function for both securitized and non-securitized mortgages, we show that for a given valuation, a bank holding a non-securitized mortgage bids more than a bank holding a securitized mortgage. This is consistent with asymmetric information being more salient for non-securitized mortgages.

%This information asymmetry has potentially important implications for securitization policies. As we explain, these policies have traditionally focused on the probability of default, in particular as measured by the borrower's FICO score. However, the total cost of a defaulted mortgage will also include the loss given default. The latter is determined by the value of the house recovered in the foreclosure auction. It is this total loss that gets insured by the US federal mortgage agencies, with the cost ultimately borne by the taxpayer. We find that (i) the bidding behavior of the lenders is consistent with the hypothesis that they are likely to be better informed when the mortgage is not securitized, and (ii) the recovery ratio is substantially lower in the pool of the securitized mortgages. Taken together, these findings suggest that some of the original lender's informational advantage concerning the recovery value of the mortgage is already present at the securitization stage. This information is likely to be private, leading to a previously unrecognized dimension of adverse selection at the securitization stage.

%The first finding is important because it indicates that the banks are likely to have \emph{private} information concerning the loss given default already at the securitization stage, effectively transferring this loss to the taxpayers. The second finding is important because it indicates that this information is likely to be privately known by the bank. This points towards a potentially important and previously unrecognized market failure at the securitization stage, and the potential need for a policy response to address it.


%There is also a generally held suspicion that lower quality
%mortgages are securitized and sold on the market, despite the fact
%that securitization agencies screen based on publicly observable
%characteristics. The suspicion is that securitized mortgages are worse
%with respect to characteristics which cannot be observed
%publicly. However, it is in the nature of characteristics that cannot
%be publicly observed that it is difficult to make empirical statements
%concerning them. With our theory giving clear predictions on what one
%should observe with and without a common value component, we can look
%in more detail at the question whether for non-securitized mortgages
%banks indeed have private information of the quality of the house. The
%data for securitized mortgages seems to be more consistent with an
%independent private values setup, whereas for non-securitized
%mortgages, it is more consistent with there being a common value
%component for one part of the auctions.

%
%What are the welfare implications of common values and the associated adverse selection in foreclosure auctions? Building on the insights in \cite{cai2007reserve}, \cite{jullien2006auction} and \cite{lamy}, we show that seller bids in the foreclosure auction under common values involve a \emph{signalling premium} for bids below the judgment amount. That is, they are higher relative to what they would have been if the realtors knew the bank's information. This introduces a further distortion, on top of the usual deadweight loss stemming from the bank's monopoly position as the seller. Fro the bids above the judgment amount, on the other hand, there is no distortion, as the bank will act as a buyer, and the English auction will allocate the house efficiently.


%
%
%
%Under
%the assumption that for non-judicial foreclosures the same buyers with
%the same valuations participate in an auction, we can make a welfare
%comparison between judicial and non-judicial foreclosures. Under an
%independent private values setup, judicial foreclosures generate
%higher welfare than non-judicial foreclosures. A comparison in a
%common values setup is work in progress.
%
\paragraph{Related Literature.} To the best of our knowledge, this is the first paper providing an economic analysis of the bidding behavior in foreclosure auctions. So far, the details of foreclosure auctions have been mainly analyzed in the law literature (see e.g. \cite{nelson2004reforming}).

For our theoretical analysis, we build on two strains of literature. The first analyzes bidding behavior of buyers with information about a common value component, starting with \cite{milgrom1982theory}. The second, more recent, literature analyzes auctions in which the seller has superior information, see \cite{jullien2006auction}, \cite{cai2007reserve}, \cite{lamy}. For bids above the judgment amount, we can use results from the former; for bids sufficiently below the judgment amount, we can use results from the latter literature, in particular from \cite{cai2007reserve}. Our theoretical contribution is to show that there are surprising predictions for the intermediate range of bids, which are between the informed seller and the informed buyer regions.

Evidence of strategic behavior and asymmetric information has long been a focus of the empirical auction literature, beginning with \cite{hendricks1988empirical} and \cite{hendricks2003empirical} for offshore oil auctions, and  \cite{nyborg2002bidder} and \cite{hortaccsu2012valuing} for treasury auctions. \cite{hortaccsu2012valuing} find evidence of dealers' informational advantage in Canadian Treasury auctions.

The application of our theory to securitization relates to the growing literature that has been sparked by the interest in the determinants of the financial crisis, such as \cite{keys2008did}, \cite{brunnermeier2009deciphering}, \cite{tirole2011illiquidity}. Our analysis confirms the suspicion voiced in this literature that securitization led to moral hazard. Our analysis can be seen as complementary to the analysis in \cite{keys2008did}, who provide evidence for asymmetric information with respect to the probability of default. Our analysis shows existence of asymmetric information with respect to the other determinant of the expected shortfall of a mortgage: the loss given default (which is determined by the value of the house).
In a wider sense, our article relates to the growing literature that uses insights from auctions to analyze important questions  in financial markets, such as \cite{heller1998auctions}, \cite{hortacsu2010mechanism}, 
\cite{cassola20132007},  \cite{zulehner2013competition}.


\section{Foreclosure process}
Property foreclosure is a remedy allowed by law to the lender if the borrower defaults on the mortgage. While it generally transfers the ownership of the property to the lender, the process may be a lengthy one and the details depend on the jurisdiction. In the US, roughy one third of the states adopt what is called a \emph{judicial} foreclosure, while the rest is comprised of the \emph{nonjudicial} foreclosure states. See Table \ref{tbl:judicial}.

The main legal difference between judicial and nonjudicial foreclosure is that in the former, the process takes place in the court system, while in the latter, it takes place outside the courts. Judicial foreclosures always result in the property being sold at a public auction, while the nonjudicial foreclosure usually involves a sale of the property by the trustee under the power of sale clause. The trustee sale may or may not be conducted through auction, but still have to generally comply with various state laws, even though it is performed outside of the court system.\footnote{A handful of states (Connecticut, Maine and Vermont) allow for a \emph{strict foreclosure}, where the property title is transferred to the lender immediately after the default.} 

\begin{table}[t!]
\centering
\begin{tabular}{c c} 
 \hline
Judicial states & Nonjudicial states  \\ [0.5ex] 
 \hline
 Connecticut & Alabama  \\ 
 Delaware & Alaska  \\
 Florida & Arizona \\
Hawaii & Arkanzas  \\
 Illinois & California  \\
 Indiana & Colorado \\
 Iowa & District of Columbia\\
 Kansas & Georgia \\
 Kentucky & Idaho \\
 Louisiana & Maryland\\
 Maine & Massachusetts \\
 New Jersey & Michigan\\
 New Mexico & Minnesota\\
 New York & Mississippi\\
 North Dakota & Missouri\\
 Ohio & Montana\\
 Oklahoma & Nebraska \\
 Pennsylvania & Nevada \\
 South Carolina & New Hampshire\\
 South Dakota & New Mexico\\
 Vermont & North Carolina\\
 Wisconsin & Oklahoma\\
 & Oregon \\
 & Rhode Island \\
 & South Dakota \\
 & Tenessee \\
 & Texas \\
 & Utah \\
 & Vermont \\
 & Virginia \\
 & Washington \\
 & West Virginia \\
 & Wyoming \\
  [1ex] 
 \hline
\end{tabular}
\caption{Judicial and nonjudicial foreclosure states. Some states use both types of foreclosure and are listed in both columns.}
\label{table:1}
\end{table}

In this paper, we focus on judicial foreclosures. In our empirical application, we study foreclosure auctions in a judicial state (Florida). The judicial foreclosure process generally consists of the following steps. 

\begin{enumerate}
\item The mortgage holder misses several payments on the mortgage. In some states, the lender is allowed to start the foreclosure process after the borrower has missed one payment. In practice, however, the process usually begins after three payments have been missed.
\item The lender informs the mortgage holder, usually through a letter of intent with a specified deadline, that it intends to begin the foreclosure process. This notice of intent, or ``breach letter", allows the borrower to make up the missed payments before the deadline. The deadline is usually set at 30 days.
\item If the mortgage is still delinquent, the lender proceeds to file a lawsuit, usually in the county where the property is located. The complaint is served to the mortgagee, usually by the county sheriff. At this point, the mortgagee becomes a defendant in the foreclosure lawsuit.
\item The defendant is given a certain amount of time, usually 20 - 30 days, to respond. The defendant does not have to  respond. The defendant may choose to respond if he or she is able to raise a substantive complaint, e.g.\ concerning the ownership of the mortgage or unfair lending practices.
\item If the complaint goes uncontested or the defendant does not raise a substantive complaint, the lender is granted a judgement of foreclosure. The judgement will specify the date of the foreclosure sale. The sale is conducted through a public auction. The bank is allowed to credit bid up to the debt amount (also called the judgement amount), while other participants (usually real estate brokers) are required to submit bids in cash or cash equivalent. The auction is usually conducted in an open format. Some counties have adopted Internet auctions.\footnote{This is the case for Palm Beach county in Florida, the source of the data for the empirical application in our paper.} The property title is awarded to the high bidder, and the auction price is equal to the highest bid. If the lender is \emph{not} the high bidder, the lender receives the payment equal to the minimum of the auction price and the judgement value. If the sale price exceeds the judgement value, the surplus is used to satisfy the junior liens, if any. The remaining surplus is paid to the defendant. 
\end{enumerate}


In the next section, we present a stylized model of a foreclosure auction that captures the main institutional features described above.


\section{Model}

Consider the owner, the bank (also called the seller, $S$) and $n$ real estate brokers
(or buyers, $B$) who participate in the foreclosure auction. The
judgment amount (i.e.\ the balance of the mortgage) is
denoted as $v_J$. The foreclosure auction is modelled as variant of the (button) English auction as in \cite{milgrom1982theory}. The auction website expedites bidding by allowing the participants to employ automatic bidding agents. The bidders provide their agents the maximum price they are willing to pay (their \emph{dropout} price), and the agent then bids on their behalf. These proxy bids can be updated at any time.\footnote{Such proxy bidding makes the button model even more applicable here, as it alleviates the need to model difficult features such as e.g.\ jump bidding that may be present in the traditional open auctions.} The brokers are able to observe if the bank's maximum bid has been exceeded, while a broker's dropout price is unobservable to the bank. The broker may therefore change its maximum bid upon observing the bank's dropout price. As for the bank, we assume that it commits to its dropout price.

The key difference from a standard auction is that the proceeding of the foreclosure auction up to the judgment amount goes to the bank, anything above the judgment amount goes to the original owner. This effectively turns the bank into a seller for prices below and into a buyer for prices above the judgment amount. 

The winner pays the auction price $p$.  If the price
exceeds the judgment amount, $p\geq v_J$, then the bank gets $v_J$ and
the owner pockets the difference $p - v_J$. If the price is below the
judgment amount, $p<v_J$, then the banks gets $p$ and the owner gets
nothing. The property is transferred to a broker only if a
broker wins; otherwise, the bank keeps the property. It follow that
if the bank wins the auction, it effectively pays the auction price to
itself, so in reality no money changes hands in this case. But if a
broker wins, then there is an actual money transfer, from the broker
to the bank and possibly the owner as well (if the auction price
exceeds the judgment amount).

As is usual, we model the foreclosure sale (auction) as a game of
incomplete information. 
As is explained in the empirical section of the paper, banks and brokers buy houses for different purposes: banks mostly sell the houses later on. Brokers typically renovate the property before reselling. Motivated by this, we make the following assumption concerning the information of the bank and the brokers. First, we assume that the $i$th broker's idiosyncratic signal, denoted as $X_B^i$, only concerns its renovation value added to the house. Second, we assume that the bank's signal $X_S$ concerns the baseline resale value of the house. The signals $X_B^i$ and $X_S$ will be sometimes referred
to as buyers' and seller's \emph{types}. Their realizations will be denoted as $x_B^i$ and $x_S$, respectively. 

% Given the different nature of the bank's and broker's signals, it is assumed that  $x_B^i$'s and $x_S$ are independent.
The bank is
assumed to be the informed party. Its signal $X_S$ is normalized to
equal the expected value of the house in the market, so the bank's
valuation is 
\[ 
u_S(x_S) = x_S .
\] 
The brokers do not observe 
$X_S$; they only privately observe their own signals $X_B^i$. Broker $i$'s expected value of the house, given its own signal $x_B^i$ and the bank's signal $x_S$, is denoted as $ u_B(x_B^i,x_S).$


%To capture the fact that input prices may be correlated across the brokers, we allow the signals $x_B^i$ to be correlated.
%
%%They, however, have information about an additional
%%idiosyncratic component of their valuations, which is incorporated in the broker's signal $x_B^i$. 
%%$x_B$ can be
%%thought of as the renovation value of the house. 
%
%We make the following
%assumptions about the buyer valuations. We assume that the
%distributions $F_B$ and $F_S$ are potentially different. The reason is
%that banks and brokers buy houses for differentnt purposes: banks mostly
%sell the houses later on. Brokers typically renovate the property
%before reselling.



We make the following assumptions concerning the expected valuations of the brokers.

\begin{assumption}[broker valuations]\label{as:valuations}
A broker's expected valuation is differentiable and strictly increasing in own signal $x_B$, and nondecreasing in the bank's signal $x_S$, 
\begin{align}
  \frac{\partial u_B(x_B,x_S)}{\partial x_B}\geq \alpha,\quad
  \frac{\partial u_B(x_B,x_S)}{\partial x_S} \geq 0, \nonumber
\end{align}
for some constant $\alpha>0$.
Moreover, the derivatives satisfy the \emph{single crossing} condition 
\begin{align}
  \frac{\partial u_B(x_B,x_S)}{\partial x_B}>\frac{\partial u_B(x_B,x_S)}{\partial x_S} \nonumber
\end{align}

%and 
%\begin{align}
%  u_B(x,x) = x.\label{eq:normalization}
%\end{align}
\label{as:utilities}
\end{assumption}




This assumption ensures that a broker's valuation of the house is
increasing in its own signal $x_B$, and is
non-decreasing in the bank's signal $x_S$. If $u_B$ does not
depend on $x_S$, we have a special case of \emph{private values}. Otherwise, the valuations are interdependent. 



For reasons that will be clear in the sequel, we normalize the broker signals so that the value conditional on winning the auction is equal to the signal,
\begin{align}
u_B(x_B,x_B) = x_B.
\label{bnorm}
\end{align}
This normalization is without loss of generality because Assumption \ref{as:valuations} ensures that $u_B(x_B,x_B)$ is continuous and strictly increasing in $x_B$.

We make the following assumptions regarding the distribution of the signals.
\begin{assumption}[Signals]\label{as:info}
The bank's signal $X_S$ is drawn from a distribution $F_S$ supported on $\mathbb R_{+}$, with density $f_S$ continuous and positive on the support. The broker signals $X_B^i$, $i=1,...,n$, are identically and independently distributed and drawn from the distribution $F_B$, supported on $\mathbb R_{+}$, with density $f_B$ continuous and positive on the support.
\medskip
\end{assumption}
%We assume that the signals are independent.
%\begin{assumption}[Independence]
%\noindent 
%\end{assumption}
The independence assumption is made to simplify the analysis of the game, by eliminating the need to consider adjustments that brokers would otherwise make to their proxy bids following dropouts by other brokers. Under independence, we shall see that the information in the auction will be transmitted only from the bank to the brokers, following the bank's dropout from the auction. After that, the brokers would essentially have independent private values, and simply enter those values as their (updated) proxy bids, and there will be no updating from brokers' dropout prices.\footnote{Independence is also assumed in \cite{jullien2006auction}, while \cite{cai2007reserve} and  \cite{lamy} allow the buyer signals to be correlated, but still independent of the seller's signal.}

The \emph{Myerson virtual value} is defined in the present setting as 
\begin{align} J_B(x_B,x_S) =
  u_B(x_B,x_S) -\frac{\partial u_B(x_B,x_S)}{\partial x_B}
  \frac{1-F_B(x_B)}{f_B(x_B)}.
  \end{align}
We make the standard monotonicity assumption concerning $J_B(x_B,x_S)$.
\begin{assumption}[Virtual value monotonicity] \label{as:myerson} The function $J_B(x_B,x_S)$ is strictly increasing in $x_B$.
\end{assumption}

In the following, we will use $x_{(1)}$ and $x_{(2)}$ for the highest and second highest order statistics among $n$ brokers' signals. Further, we will use $F_{(1)}$ and $F_{(2)}$ for the corresponding distributions.

We have to reasons to focus on a model in which the seller's superior information about the quality of the property being auctioned is at the center of interest. First, the bank is likely to have better information due to having gathered information about the property when granting the mortgage. Information about the mortgagee can also serve as an indication of how well the mortgagee maintains the house. Second, we will later consider securitized and non-securitized mortgages. As it will become clearer later on, there are reasons to believe that the informativeness of the bank's signal about the quality of the house is different for securitized and non-securitized mortgages.

\section{Symmetric information}
\label{sec:indep-priv-valu}

It is useful to first start with the case of \emph{symmetric information}, where the bank's information is known to the brokers.  This independent private values (IPV) setup is particularly useful to highlight the role of
the judgment amount, below which the bank acts as a seller and above
which the bank acts as a buyer.

%Under IPV, our normalization implies \[u_B(x_B,x_S) = x_B,\] and the Myerson virtual value here is simply \[ J_B(x_B) = x_B -   \frac{1-F_{B}(x_B)}{f_{B}(x_B)}. \]
Standard arguments imply that it is a weakly dominant strategy for the broker to
choose its valuation as the drop out price, so 
\[p_B(x_B) = x_B.
\] 
%\begin{remark}
%If buyer signals are independent, or, by convention, if there is only one buyer, $J_B(\cdot)$ becomes the standard Myerson virtual value, \[ J_B(x_B) = x_B -   \frac{1-F_{B}(x_B)}{f_{B}(x_B)} .\] 
%\end{remark}

The bank's bidding behavior can be best derived by first considering two hypothetical cases. First, nothing is owed to the bank ($v_J=0$) and hence the bank always acts as a buyer. Second, an infinite amount is owed to the bank ($v_J=\infty$) and hence the bank always acts as a seller. After deriving these two hypothetical cases, we can put the two pieces together and additionally derive the bank's bidding behavior for the transitional region in which the bank turn from a seller to a buyer.

First, consider the case in which the bank always acts as a buyer ($v_J=0$). By standard arguments for English auctions, the bank's optimal strategy is to bid its own type, i.e. $p_S(x_S)=x_S$.

Next, consider the case when $v_J = \infty$, i.e.\ the standard auction where the bank acts as the seller. The brokers will drop out at prices $u_B(x_B^i,x_S)$. If the bank decides to drop out at price $p$, its auction revenue will be equal to $p$ if there is one, and only one broker that is active in the auction at that price. If there are multiple brokers active at $p$, the bank's revenue will be equal to the second-highest dropout broker price, i.e. $u_B(x_{(2)},x_S)$. Denote as $\hat x_B$ the broker's type indifferent between staying in or dropping out at $p$, so that $p = u_B(\hat x_B, x_S)$. Then the bank's expected profit is equal to 
\[ \Pi_S(x_S, \hat x_B) = u_B(\hat x_B,x_S) n F_B(\hat x_B)^{n-1} (1-F_B(\hat x_B)) 
+\int_{\hat x_B}^\infty u_B(y,x_S) f_{(1)}(y) dy
+x_S F_B(\hat x_B)^n.
\]
(If there is only one broker, $n=1$, then the integral term should be excluded from the above formula.) The bank will choose the price, or, equivalently, the broker's marginal cutoff $\hat x_B$, optimally. The derivative of $\Pi (x_S,,\hat x_B)$ with respect to $\hat x_B$ is equal to
\[ \frac{\partial \Pi (x_S,\hat x_B)}{\partial \hat x_B} = -nf_B(\hat x_B) F_B(\hat x_B)^{n-1} \Big ( J_B(\hat x_B,x_S) - x_S \Big)
\]
for any $n\geq 1$. Assumption \ref{as:myerson} implies that $\Pi (\hat x_B,x_S)$ is \emph{quasiconcave} in $\hat x_B$, and is uniquely maximized at $x_B^*(x_S)$ that satisfies the FOC 
\begin{align} J_B(x_B^*(x_S),x_S) = x_S . \label{eq:sc} \end{align}
This is, of course, a well-known result in auction theory concerning the optimal reserve price, adapted to the setting where the buyer's valuations depend on the seller's information $x_S$, observable to the buyers.\footnote{See \cite{myerson1981optimal}.}  In general, the bank's optimal strategy 
\begin{align}
p_S^0(x_S) = u_B(x_B^*(x_S), x_S) \label{eq:ps0-below-vJ}
\end{align}
may be non-monotone in $x_S$. Totally differentiating \eqref{eq:sc} with respect to $x_S$ yields
\[ \frac{d x_B^*(x_S)}{dx_S} = \frac{1-\partial J_B/\partial x_S}{\partial J_B/\partial x_B} \]
So a sufficient condition for the monotonicity is \begin{align}
\frac{\partial J_B(x_B,x_S)}{\partial x_S} < 1 \label{eq:cond1}
\end{align}
It can be checked that \eqref{eq:cond1} is implied by the stronger but more intuitive conditions
\begin{align}
\frac{\partial u_B(x_B,x_S)}{\partial x_S} <1,\quad \quad \frac{\partial^2 u_B(x_B,x_S)}{\partial x_B \partial x_S } \geq 0.
\label{eq:cond2}
\end{align}

Now consider the case of our primary interest, $0<v_J<\infty$. If the bank has valuation $x_S \leq v_J$, it will not drop out at the the price above $v_J$: the bank is not entitled to any revenue in excess of $v_J$, so staying in the auction beyond $v_J$ will only reduce the probability of sale. So effectively, the bank's problem is to maximize its profit with the additional constraint $p \leq v_J$, or, equivalently, $u_B(\hat x_B,x_S) \leq v_J$. As we have seen, the bank's profit is quasiconcave in $\hat x_B$, so the optimal price is given by 
\[ 
p_S^0(x_S)= \begin{cases}
u_B(x_B^*(x_S),x_S) & \text{if $x_S<\ul x_S$,} \\
v_J & \text{otherwise,}
\end{cases}
\]
where the border between the separating and the bunching region $\ul x_S$ is implicitly define by $u_B(x_B^*(\ul x_S),\ul x_S)=v_J$.

If, on the other hand, $x_S>v_J$, then the bank is not willing to sell the house at the price below $x_S$, but also will not benefit from a price above $x_S$ as the excess $p - v_J$ will go to the owner. It follows that the bank's optimal strategy in this case is to drop out at $x_S$,  $p_S(x_S) = x_S$. 

We summarize these findings in the proposition below. Refer to Figure \ref{fig:equilibrium}.
\begin{proposition}\label{prop:syminfo}
The bank's optimal strategy is given by 
\[ p_S^0(x_S) = 
\begin{cases}
u_B(x_B^*(x_S),x_S) & \text{if $x_S<\ul x_S$,} \\
v_J & \text{if $x_S\in[\ul x_S,v_J]$} \\
x_S, & \text{if $x_S > v_J$}
\end{cases}
\]
The optimal strategy is strictly increasing in $x_S$ provided \eqref{eq:cond1} or \eqref{eq:cond2} hold, in which case the bank's prices are pooled over an interval $[\underline x_S, v_J]$.
\end{proposition}
 
%
%With these preliminaries, the bank's dropout strategy is characterized
%in the proposition below. 
\begin{figure}[t]
\centering
\includegraphics[scale=0.55]{graphics/equilibrium.pdf}
\caption{Equilibrium under IPV\label{fig:equilibrium}}
\end{figure}

\begin{figure}[t]
\centering
\includegraphics[scale = 0.5]{graphics/model_equilibrium}
\caption{Computed 
	equilibrium bid distributions for the bank
  ($G_S(\cdot)$; solid curve) and the broker ($G_B(\cdot)$; dashed
  curve). It is assumed $u_B(x_B,x_S) = x_B$, and the value distributions are specified as normal, with equal
  means $\mu_S = \mu_B = 0.4$ and standard deviation $\sigma_S =
  \sigma_R = 0.6$. The judgment amount is $v_J=1$. Observe the mass
  point at the judgment amount in
  $G_S(\cdot)$.\AN{It's probably better to only plot the bank's bid distribution.}\label{fig:model_equilibrium}}
\end{figure}

\paragraph{Discussion} The bank's equilibrium
behavior is different depending on whether the bank's valuation is
below or above the judgment amount. The bank acts in a seller role in
the former case, and in a buyer role in the latter. The most
interesting feature of the equilibrium is the bunching region. In this region, the bank's dropout prices are pooled
at the judgment amount (see Figure \ref{fig:equilibrium}).

The following intuition can be given for the bunching of banks' bids at the judgment amount. Consider comparative statics with respect to the banks valuation $x_S$. For low valuations $x_S$, the bank bids its valuation plus the monopoly markup. The optimal bid of the bank balances two effects in a trade-off: a higher bid increases the bank's expected profits conditional on selling, but also lowers the probability of selling the house to a third party. As we increase $x_S$, this trade-off changes in a way that makes higher bids more attractive to the bank. Hence, the banks bid increases with $x_S$. However, at the point where the bank's bid reaches $v_J$, one part of the trade-off disappears for price increase (but not decreases): a higher bid by the bank does not increase the bank's profit conditional on selling, since any additional revenues go to the original owner. Therefore, as $x_S$ increases, the bank's optimal bid stays at $v_J$. Once $x_S$ surpasses $v_J$, the bank actually prefers retaining the property, since it is worth more than the amount owed. Therefore, the bank's bid will again increase with $x_S$ for $x_S$ above $v_J$.
%Why are the bank's dropout prices bunched at the judgment amount?
%To get an intuition for this, note that the bank will only benefit from a
%dropout price above $v_J$ if it keeps the house, i.e.\ wins the
%auction. Otherwise, if it sells the house, then even though the sale
%price would be higher than $v_J$, the bank stands to pocket only the
%judgment amount $v_J$. So if the bank's valuation is \emph{below}
%$v_J$, the bank would prefer to take $v_J$. This explains why there is
%a constraint $p\leq v_J$ in the bank's optimal pricing problem when
%$x_S \leq v_J$, which causes bunching at $v_J$ for bank's values
%somewhat below $v_J$.
%
%This bunching will cause a \emph{mass point} at $v_J$ in the
%distribution of the bank's dropout
%prices. 
This flat piece in the bank's bidding function corresponds to a mass point in banks' bid distributions.
Figure~\ref{fig:model_equilibrium} illustrates this in an
example with normal distributions.

 
\section{Asymmetric information}
\label{sec:asymmetric-information}

We now consider the general environment with common values, where the bank will be the informed party. As we shall see, there are some novel features in this setting compared to the symmetric information setup. The main new feature is that the equilibrium will involve a \emph{gap} below the judgment amount. This should be intuitive since otherwise the bank would prefer to deviate from prices slightly below $v_J$ to offering $v_J$, leading both to a higher average price and not changing the probability of sale.

As with symmetric information, the bank will act as a seller for lower types, and as a buyer for higher types. Again, we begin by considering the extreme cases $v_J = 0$, where the bank always acts as a buyer, and $v_J = \infty$, when the bank always acts as a seller. 

\subsection{Bank-buyer equilibrium ($v_J = 0$)}
\label{sec:vj0}

If $v_J=0$, both the bank and the brokers act as buyers, bidding in a standard English auction.\footnote{We follow \cite{milgrom1982theory} and assume the \emph{thermometer} model, where the price is risen continuously from $0$ until there is only a single buyer left.} The bank knows its valuation $x_S$, so will dropout at the price equal to $x_S$. In a symmetric equilibrium,\footnote{By symmetric here we mean that the \emph{brokers} adopt the same strategy.} a broker's dropout strategy $p_B(x_B)$ is found through the standard indifference condition. The intuition is that, given the auction has reached price $p$, if a broker decided to drop out at price $p$ instead to $p+\epsilon$, this decision will change the broker's expected payoff only if the bank (and the other brokers) dropped out at prices $p \in (p,p+\epsilon)$. If the bank drops out at $p = x_S$, then in the limit as $\epsilon \to 0$, the expected value of the house to the broker will be $u_B(x_B, p)$. In equilibrium, the broker will drop out at a price such that he is indifferent between dropping out and continuing, while single crossing implies that brokers with higher types will prefer to continue. Thus, a broker's type $X_B(p)$  dropping out at price $p$ is (uniquely) found from the condition \[ u_B(X_B(p),p) = p.\] Given our normalization $u_B(x,x) = x$, we see that $X_B(p) = p$, so the broker, even though uninformed, in equilibrium will also drop out at the price equal to its type. This equilibrium is described in the proposition below. 

\begin{proposition}[Bank-buyer equilibrium] 
In a realtor-symmetric equilibrium, both the bank and the brokers will drop out at the prices equal to their signals, \[ p_S(x_S) = x_S, \quad \quad p_B(x_B) = x_B. \]
\end{proposition}

\subsection{Bank-seller equilibrium ($v_J = \infty$)}

\label{sec:comm-value-comp}




%
%
% 
%\subsection{Non-judicial foreclosures}\label{sub:nonjudicial}
%
%Recall that in a non-judicial foreclosure, the bank is entitled to the entire amount of the winning broker's bid. So we model a non-judicial foreclosure is simply an English auction where both the bank and the brokers are strategic participants.
%This setting essentially involves a secret reserve price set by the informed seller (the bank) in a standard English auction. We will refer to this setting as the \emph{bank-seller equilibrium}.
Auctions in which the seller has information about the common value
component have been considered in
\cite{jullien2006auction}, \cite{cai2007reserve} and \cite{lamy}.\footnote{These papers consider the case of a publicly observable seller's reserve price in a second-price auction.} 
These papers consider the case of a public reserve price and 
characterize a separating equilibrium in strictly increasing,
continuous and differentiable strategies.  
%In the sequel, we shall
%refer to this equilibrium as an \emph{unconstrained} equilibrium. 
%In 
%the foreclosure auction, the bank's revenue is constrained not to
%exceed the judgment amount.
%In our dataset, there are some auctions with a public reserve. But in the majority of the auctions, the bank is an active participant. In other words, the bank bids just like any other bidder. 
We begin by adapting the equilibrium characterization results in the aforementioned papers to our setting.



We restrict attention to equilibria where the bank adopts an increasing and continuous equilibrium dropout strategy $p_S^*(x_S)$, with a differentiable inverse $X_S^*(p)$. Given our assumption that broker signals are independent, only the bank's dropout price is relevant for information updating. Denote a broker's dropout strategy as $p_B^*(x_B)$, with the inverse denoted as $X_B^*(p)$. As in \cite{milgrom1982theory}, $X_B^*(p)$  is found by equating the object's expected value to the broker assuming the bank drops out at $p$, to the price $p$:
\begin{align}
u_B(X_B^*(p),X_S^*(p)) = p  \label{eq:ubgen} 
\end{align}
Following the bank's dropout at a price $\tilde p$, a broker's dropout strategy is simply $u_B(x_B,X_S(\tilde p))$ as the brokers will then have independent private values.







%In parallel to Assumption \ref{as:myerson} in the previous section, we assume that $J_B(x_B,x_S)$ is monotone is the buyer's signal.\footnote{In the independent private values (IPV) case, $ u_B(x_B,x_S)  = x_B$ according to our normalization, and one can show $ \frac{F_{(2)}(x_B)-F_{(1)}(x_B)}{f_{(1)}(x_B)} = \frac{1-F_B(x_B)}{f_B(x_B)}$. So $J_B(x_B,x_S)$ becomes the usual Myerson virtual value as in the previous section.}
%\begin{assumption}[Virtual value monotonicity] \label{as:virtual_value}
%  The function $J_B(x_B,x_S)$  is increasing in $x_B$.
%\end{assumption}
The following proposition describes the separating equilibrium in our model.


\begin{proposition}[Bank-seller equilibrium]\label{prop:nonjudicial}
  There is a unique equilibrium in monotone differentiable
  strategies. The bank's and the broker's 
  inverse bidding strategies $X_S^*(p)$ and $X_B^*(p)$ are given by the (unique)
  solutions to the differential equations 
  \begin{align}
 \frac{d X_S^*(p)}{dp}&= 
  \frac{(J_B(X_B^*,X_S^*)-X_S^*)]f_{(1)}(X_B^*)}{
  \frac{\partial u_B}{\partial x_S}
   (u_B(X_B^*,X_S^*)-X_S^*)f_{(1)}(X_B^*)+
   \frac{ \partial u_B}{\partial x_B }
    \int_{X_B^*}^\infty \frac{  \partial u_B} {\partial x_S} 
    f_{(2)}(x) dx
    },\label{eq:sdifeq}
    \\ \nonumber
    \\
  \frac{d X_B^*(p)}{dp}&= 
  \frac{ \frac{  \partial u_B} {\partial x_S}( F_{(2)}(X_B^*) -  F_{(1)}(X_B^*)) +  \int_{X_B^*}^\infty \frac{  \partial u_B} {\partial x_S} 
    f_{(2)}(x) dx  }{
  \frac{\partial u_B}{\partial x_S}
   (u_B(X_B^*,X_S^*)-X_S^*)f_{(1)}(X_B^*)+
   \frac{ \partial u_B}{\partial x_B }
    \int_{X_B^*}^\infty \frac{  \partial u_B} {\partial x_S} 
    f_{(2)}(x) dx \label{eq:bdifeq}
    }   ,
\end{align}
%  \eqref{eq:sdifeq} and \eqref{eq:bdifeq},  
subject to the initial conditions $X_S^*(\ul p) =
  0$ and $X_B^*(\ul p) = \ul p$.  The lowest price offered by the
  bank $\ul p$ is given by $ \ul p = p_S(0) = u_B(\ul x_B, 0)$, where $\ul x_B$  is the lowest broker type that  purchases with positive probability, given by the unique
  solution to $J_B(\ul x_B,0) = 0$.
For an out-of-equilibrium reserve price $p<\ul p$, brokers believe
  that the bank's type is the lowest possible.
\end{proposition}
We provide the proof in the Appendix. The proof adapts  \cite{cai2007reserve} to our setting where the bank is an active bidder in the open auction. We also show that the full-information price is the only price that can be offered by the lowest-type bank in such a separating equilibrium. Thus the equilibrium outcome is unique  (assuming that the strategies are differentiable and monotone). The out-of-equilibrium beliefs for prices lower than $\underline p$ are indeterminate, but must be sufficiently pessimistic so as to provide the bank with an incentive not to drop out at lower prices. The most pessimistic beliefs (i.e.\, believing that the bank's type is the lowest possible) are reasonable, and they indeed support this unique equilibrium outcome.



%\begin{remark}
%\cite{cai2007reserve} obtain a parallel result for the case of a public reserve. It turns out that the bank's equilibrium reserve price strategy is the same as the dropout strategy above, and the brokers' dropout strategy is also the same as in their paper. The intuition is that $X_B(p)$,  the buyer's type who drops out at price $p=p_S(x_S)$ when the bank bids in the auction  is the same as the marginal participating buyer type at $p_S(x_S)$ when the reserve price is public. So if $p_S(x_S)$ is an equilibrium bank's dropout strategy in the open auction, it will also be an equilibrium reserve price strategy when the reserve is public.
%\end{remark}

There is a noteworthy property of this separating equilibrium, in comparison with the symmetric information setup considered in Section \ref{sec:indep-priv-valu}, the \emph{signalling premium}. 
\begin{corollary}[Signalling premium]\label{cor:sp}
Under asymmetric information, the bank bids higher, $p_S^*(x_S) > p_S^0(x_S)$ for $x_S>0$.
\end{corollary}

\begin{proof} Our previous analysis in that section implies that the bank with valuation $x_S$ will set the price so that the marginal broker type willing to purchase at this price, $X_B^0(p)$, is found from the ``marginal revenue equals cost" equation $J_B(X_B^0(p),x_S)-x_S=0$. The price strategy itself is given by $p_S^0(x_S) = u_B(X_B^0(p),x_S)$. How does this price compare to the one with asymmetric information, $p_S^*(x_S)$? Proposition \ref{prop:nonjudicial} shows that there is no distortion at the bottom, so that the two price are equal: $p_S^*(0) = p_S^0(0)$.  

For $x_S>0$,  we have $d X_S^*(p)/dp>0$. Going back the differential equation \eqref{eq:sdifeq}, this means that for $p=p_S^*(x_S)$,  we must have $J_B(X_B^*(p),x_S)>x_S$, which, by the monotonicity of $J_B(x_B,x_S)$ in $x_B$, implies $ X_B^*(p)>X_B^0(p)$. Since $u_B(X_B^0(p),X_S^0(p)) = p$ and $u_B(X_B^*(p),X_S^*(p)) = p$, with $u_B(x_B,x_S)$ being monotone increasing in both arguments (under common values), we must have \[ X_S^*(p)<X_S^0(p) \implies p_S^*(x_S) > p_S^0(x_S)\] for $x_S>0$. \end{proof}



The signalling premium leads to welfare losses under asymmetric information. Consider, for simplicity, the case of a single broker. Our normalization $u_B(x_B,x_B) = x_B$ implies that it is efficient to transfer the object from the seller with valuation $x_S$ to the buyer with valuation $x_B$ if and only if $x_B>x_S$. Denote as $x_B(x_S)$ the minimal buyer type that will, in equilibrium, trade with the seller with valuation $x_S$. It is found from $u_B(x_B(x_S),x_S) = p(x_S)$. Under asymmetric information, this type is determined from $u_B(x_B^*(x_S),x_S) = p_S^*(x_S)$, while under symmetric information, it is determined from $u_B(x_B^0(x_S),x_S) = p_S^0(x_S)$. Since $p_S^*(x_S)>p_S^0(x_S)$ due to the signalling premium, we must have \[ x_B^*(x_S)>x_B^0(x_S)>x_S .\]
Thus, the trading boundary, already distorted even under symmetric information due to the market power of the seller, is distorted even further under asymmetric information.

Motivated by our empirical application, we now investigate how the bank's strategy is affected as its information concerning the common value component becomes less precise, and hence less relevant to the brokers. Following  \cite{cai2007reserve}, we consider a linear specification in the form
\[ u_B(x_B,x_S) = x_B+\alpha x_S, \quad\quad \alpha \in [0,1).\]
Here, $\alpha$ reflects the relevance of the bank's information for the broker.\footnote{This specification does not satisfy our normalization $u_B(x_B,x_B) = x_B$, but it will if we change the broker's signal to $\tilde x_B = \frac{x_B}{1-\alpha}$.}  As $\alpha$ decreases, the bank's information becomes progressively less relevant to the broker. The case $\alpha = 0$ corresponds to independent private values. Note that this is different from the symmetric information case considered above since now the bank's information becomes irrelevant rather than being revealed to the brokers. We denote the bank's strategy $p_S(x_S;\alpha)$.
\begin{proposition}[The effect of $\alpha$]\label{prop:slope}
In the linear model, $p_S(x_S;\alpha)$ is increasing in $\alpha$.
\end{proposition}
\begin{proof}See the Appendix
\end{proof}
%In the linear model, the lowest bank's bid $p_S(0;\alpha)$ is equal to $J^{-\mathbbm 1} (\alpha \mathbbm E x_S) - \alpha \mathbbm E x_S$, and can either increase or decrease with $\alpha$ depending on the shape of the density $f_B(\cdot)$. If $p_S(0;\alpha)$ is increasing in $\alpha$, then $p_S(x_S;\alpha) <p_S(x_S;\alpha')$ for $\alpha<\alpha'$, i.e.\ the bank's bid increases with $\alpha$ for any $x_S$. But in general the monotonicity of the slope in $\alpha$ only implies that the curves $p_S(x_S;\alpha)$ and $p_S(x_S;\alpha')$ cross at most once, with $p_S(x_S;\alpha')$ crossing from below.

\subsection{General case: $v_J \in (0,\infty)$}

In a foreclosure auction with  $v_J\in (0,\infty)$, the equilibrium will combine the features of the bank-seller equilibrium for lower $x_S$, where the bank will drop out at $p_S^*(x_S)$ as described for the $v_J=\infty$ case, and bank-buyer equilibrium for higher $x_S$, where the bank would drop out at $p_S^*(x_S) = x_S)$. However, ``stitching" these two equilibria under common values  is by no means a simple matter. To see the difficulties that arise, suppose we have a bunching equilibrium with common values where the bank types over a certain interval bid $v_J$: 
\[ p_S(x_S) = v_J, \quad\quad x_S \in [\underline x_S,\overline x_S] .\]
%Recall that in a judicial foreclosure, the
%bank is only entitled to the sale revenue up to $v_J$. As we have
%shown in the previous section, under IPV, this results in pooling of
%the seller's optimal price offers at $v_J$, while the offer below
%$v_J$ are unaffected. 

To begin, we observe that there does not exist an equilibrium in continuous strategies that involves bunching. 
%For simplicity, consider the case when there is only one broker, $n=1$. 
%Incentive compatibility implies that the probability of transfer of the house to the broker must be non-increasing in the price $p$. As the probability of transfer is equal to $1-F_B(X_B(p))$, it follows that the broker's cutoff $X_B(p)$ must be nondecreasing. 
If the bank's dropout strategy $p_S(x_S)$ were continuous and involved bunching, then the brokers' dropout strategy $p_B(x_B)$ would involve a gap below $v_J$, since otherwise the brokers who would contemplate bidding slightly below $v_J$, would prefer to wait and drop out at a price slightly above $v_J$, to take advantage of a dramatically higher quality of the house they would be able to get for banks in the bunching region. Such a gap in broker bids, however, creates an incentive for the bank types that bid somewhat below $v_J$ to deviate to the bid $v_J$, since this deviation leads to a higher expected price, but does not change the probability of selling.

  

\begin{figure}[tf]
  \centering
  \includegraphics[scale = 0.6]{graphics/eqm_common.pdf}
  \caption{Equilibrium with common values}
  \label{fig:eqm_common}
\end{figure}

In order to prevent such deviations, an equilibrium with bunching must involve a \emph{gap} below $v_J$, with $\underline x_S$, the seller at the lower bound of support of the bunching region, being indifferent between bidding $v_J$ and bidding $\ul p_S<v_J$. Sellers with valuations slightly below $\ul x_S$ will strictly prefer bidding slightly below $\ul p_S$ to bidding $v_J$.
 The bank's strategy in our (semi-)separating foreclosure equilibrium coincides with $p_S^*(x_S)$ derived previously for $v_J=\infty$, involves a jump at $\ul x_S$ to $v_J$, bunching at $v_J$, and truthful bidding for $x_S > v_J$.

The broker's dropout strategy will be a best response to the bank's strategy. It will coincide with $p_B^*(x_B)$ derived for the $v_J=\infty$ case for $x_B < x_B^*$, where $x_B^*$ is implicitly defined by $p_B^*(x_B^*)=\ul p_S$. Then, the types $x_B \in (x_B^*,\ol x_B$) will drop out as soon as the price has surpassed $\ul p_S$. These brokers realize that they cannot beat the bank profitably at any price. The types $x_B \geq \ol x_B$ will continue, and bid up to their values. We must have $\ol x_B > v_J$, since otherwise the broker bidding slightly above $v_J$ would prefer a deviation to a bid slightly above $\ul p$ in order to avoid a loss due to a lower average quality over the bunch. 

So the bank's dropout strategy $p_S(x_S)$ and the broker's dropout strategy when the bank hasn't dropped out, $p_B(x_B)$, are given respectively by
\begin{align}
p_S(x_S) & =
\begin{cases}
  p_S^*(x_S), & \quad x_S \in [0, \underline x_S], \\
  v_J,        & \quad x_S\in[\underline x_S,v_J],\\
  x_S ,    & \quad x_S>v_J,
\end{cases} \nonumber \\
p_B(x_B) & =
\begin{cases}
  p_B^*(x_B), & \quad x_B \in [\underline x_B, x_B^*), \\
 \ul p_S,        & \quad x_B\in[x_B^*,\overline x_B] ,\\
  x_B ,    & \quad x_B>\overline x_B.
\end{cases}
\label{eq:strategies}
\end{align}
See Figure \ref{fig:eqm_common}, where the bank's and broker's bidding strategies are shown respectively by the thick and the thin curves.\AN{maybe change lines to dashed and solid}



We now introduce the key two conditions that will uniquely pin down the cutoff types $\underline x_S$ and $\ol x_B$ and therefore pin down an equilibrium.
First, the $\ol x_B$-type broker must be indifferent
between dropping out at a price slightly above $\ul p$ or staying in up to $\ol x_B$. By deviating to a higher bid $\ol x_B$, the broker would gain the property value $u_B(\ol x_B,x_S)$ at a price equal to $v_J$ if $x_S \in [\underline x_S,v_J]$, and $x_S$ if $x_S \in [v_J,\ol x_B]$. So the broker's indifference condition takes the form
\begin{align}
 0 =  \int_{\underline x_S}^{\ol x_B} \Big( u_B(\ol x_B,x_S)-  \max \{ v_J , x_S \}\Big) f_S^*(x_S)d x_S  =: H(\underline x_S,\ol x_B), \label{eq:bindif0}
\end{align} 
where $f_S^*(x_S)$ is the density of the bank's types conditional on  $x_S \in [\ul x_S,\ol x_B]$, 
\[ f_S^*(x_S) = \frac{f_S(x_S)}{F_S(\overline x_B)-F_S(\underline x_S)}.\]




The second condition specifies that the $\underline x_S$-type bank is
either indifferent between bidding $\underline p_S = p_S(\ul x_S)$ or $p=v_J$ (if $\underline x_S > 0$), 
or weakly prefers the bid at $v_J$ to 
$\underline p_S$ (if $\underline x_S = 0$):
\begin{align}
  \Pi_S(\underline x_S,\underline p_S) &= \Pi_S(\underline x_S,v_J),\quad\quad & (\underline x_S > 0)
\label{eq:sindif0}
\\
 \Pi_S(0,\underline p_S) &\leq \Pi_S (0, v_J), \quad\quad  & (\underline x_S = 0) \label{eq:sindif2}
\end{align}
where $\Pi_S(x_S,p)$ is the bank's expected equilibrium profit if its type is $x_S$ and it bids $p$.\footnote{See the Appendix for the explicit formula for  $ \Pi_S(x_S,p)$.}
%There is also a possibility of a \emph{corner solution} with $x_S=0$.  It corresponds to \emph{full pooling} below $v_J$, and only requires \[ \Pi_S(0,\underline p_S) \leq \Pi_S (0, v_J) .\]

%Note that given
%$\underline x_S$ and $\underline p_S=p_S^*(\underline x_S)$, the buyer's type that is indifferent
%between buying or not at $\underline p_S$ is uniquely determined as $
%x_B^*$. 
%In this form, it is clear that \eqref{eq:sindif} is an equation in terms of $\underline x_S, \ol x_B$ only. 
%Both that, unlike \eqref{eq:bindif0}, this equation does not depend on $\gamma$. 
Our first result is a technical lemma below that shows existence of cutoffs $\underline x_S, \ol x_B$ that solve the the indifference conditions \eqref{eq:bindif0} and \eqref{eq:sindif0}. Note that it could happen that $\ul x_S = 0$, in which case the bunch extends all the way to the left.
\begin{lemma}[Existence and uniqueness of the cutoffs]\label{lm:cutoffs}
There exists a unique solution $(\underline x_S, \ol x_B)$ to the broker's and bank's indifference conditions \eqref{eq:bindif0} and \eqref{eq:sindif0}, \eqref{eq:sindif2}. 
\end{lemma}
\begin{proof}See the Appendix.\end{proof}


Given the existence of the cutoffs, we now establish existence (and uniqueness) of a semi-pooling equilibrium under asymmetric information of the kind described above.
\begin{proposition}[Equilibrium existence]\label{prop:equilibrium_existence}
There exists a unique equilibrium in the class of strategies given by \eqref{eq:strategies}.
\end{proposition}
\begin{proof}[Proof of Proposition \ref{prop:equilibrium_existence}] We begin with the bank's equilibrium strategy $p_S(x_S)$. The arguments from the previous sections imply that $p_S(x_S)$ is an equilibrium best response for $x_S \leq \underline x_S$ and $x_S \geq \overline x_B$, so in this proof we only consider $x_S \in (\underline x_S, \overline x_B)$. For $x_S \in (v_J, \ol x_B)$, the bank acts as a buyer, and it is a best response for it to bid its value, $x_S$. 
So it only remains to consider $x_S  \in (\underline x_S, v_J]$. These types will prefer to bid $v_J$ over any bid in $(\underline p_S, v_J)$. The reason is that the brokers with values $x_B<\overline x_B$ drop out immediately once the price has gone over $\underline p_S$. The brokers who remain will bid up to their values $x_B\geq \overline x_S$. The bank will prefer to bid $v_J$ over any $(\underline p_S, v_J)$ since doing so will not reduce the probability of selling to a broker, but will at least weakly increase the price conditional on sale.\footnote{The price conditional on sale will be strictly higher $v_J>\ul p$ if there is only one broker who is active, i.e.\ has $x_B\geq \overline x_S$. If there is more than one broker, the expected price conditional on sale will be unchanged.} 

Having shown equilibrium incentives for the bank, we now turn to the broker. The results in the previous section imply that, for  $x_B \leq x_B^*$, $p_B(x_B)$ dominates any other bid $p \leq \ul p_S = p_B(x_B^*)$. Since bidding $ p \in (\ul p_S,v_J) $ will not affect the broker's expected profit due to the fact that no other participant bids there, it remains to be shown that a broker with $x_B <  \ol x_B$ will not have an incentive to deviate to a $p\geq v_J$. It is clear that such a broker will not have an incentive to deviate to a bid $p> \ol x_B$, since this would lead the broker to buy at prices that are too high and would result in a loss. Indeed, by single crossing, $u_B(x_B,p) < p$ for $p>x_B$, which implies $u_B(x_B,p) < p$ for $p > \ol x_B$.

The incremental expected profit from deviation to a price $p \in [v_J,\ol x_B]$ is
\begin{align*}
\Delta \Pi_B &= \int_{\ul x_S}^{p} (u_B(x_B,x_S) - \max \{ v_J, x_S \}) f_S(x_S) dx_S
\\
&< \int_{\ul x_S}^{p} (u_B(\ol x_B,x_S) - \max \{ v_J, x_S \}) f_S(x_S) dx_S
\\
&\leq \int_{\ul x_S}^{\ol x_B} (u_B(\ol x_B,x_S) - \max \{ v_J, x_S \}) f_S(x_S) dx_S=0
\end{align*}
where the first inequality follows from the fact that $u_B(x_B,x_S)$ is increasing in $x_B$, while the second inequality follows from the definition of $\ol x_B$ as the type that is indifferent This shows that the broker with $x_B <  \ol x_B$ will not have an incentive to deviate to a $p\geq v_J$, and it also follows that the broker types $x_B \in [x_B^*,\ol x_B]$ will bunch at $\ul p_S$.
%


The  equilibrium uniqueness follows because there are unique cutoff types $\underline x_S$ and $\overline x_S$ according to Lemma \ref{lm:cutoffs}.
\end{proof}






\section{Testable Hypotheses}
\label{sec:empir-pred}

Our theory predicts bunching of banks' bids at $v_J$ for both symmetric and asymmetric information. Denoting the distribution of bank's bids as $G_S(\cdot)$, we therefore have the following testable hypothesis.
\begin{conjecture}[Bunching at $v_J$]\label{hyp:bunching}
Under both symmetric and asymmetric information, the distribution $G_S(p)$ has an atom at $p = v_J$, formally, 
\[
\lim_{p\uparrow v_J} G_S(p) <G_S(v_J).
\]
\end{conjecture}

In a hypothetical world, in which the econometrician could perfectly observe all forms of heterogeneity, it would be straightforward to test for asymmetric information: our theory predicts a gap in banks' bid distributions just below $v_J$ for asymmetric information auctions, but not for symmetric information auctions. Hence, the presence or absence of a gap could serve as a test for asymmetric information.

However, one should not expect a gap if there is unobserved heterogeneity with respect to
information asymmetry, as one certainly would expect in reality. If the sales of some of the houses are characterized by asymmetric information (ASI), whereas for others the auction is essentially under symmetric information (SI), then the SI auction will fill out the gap. Here, by SI we mean the houses where the bank's information is also available to the broker; in our model, this implies that $x_S$ is observable to the broker.



But heterogeneity with respect to informational asymmetry does have
empirically testable implications. 
%From now on in this section, we focus on an additive model where the buyer's valuation is given by
%\[ u_B(x_B,x_S) = x_B+
%Take the simplest example, in which
%there are two types of  houses, ``symmetric information (SI) houses" and ``asymmetric information (ASI) houses". 
The testable implication can be derived from Corollary~\ref{cor:sp} (the signaling premium), by observing that for ASI houses the probability of sale is lower than for SI houses for a given bid by the bank. The empirically testable implication stems from a selection due to the gap for ASI auctions as described below.

For SI houses, the bank and the brokers are symmetrically informed. In our framework, this means that the bank's information $x_S$ is observable to the broker and that there is no gap below $v_J$. For ASI houses,  $x_S$ is not observable to the broker and there is a gap in banks' bids in the interval $(\ul p_S,v_J)$ for $\ul p_S=p_S(\ul x_S)$. For bank's bids below the gap for ASI houses ($p_S<\ul p_S$), we will observe a mixture of SI and of ASI houses. The probability of sale for such prices is the
average of the high SI and the low ASI probability of sale. In the gap ($p_S\in(\ul p,v_J)$), we only observe high probability of sale SI houses. This selection effects leads to an increase of the probability of sale at $\ul p_S$. For $p_S=v_J$ and above, we again observe both SI and ASI
houses, hence a downward jump in the probability of sale at
$p_S=v_J$.
%
%Then again a sharp increase of the probability of sale just
%above $p_S=v_J$, since this in the interval $(v_J,p_2)$ more weight is put on IPV houses since ASI houses appear only when the upper ``arm" is played, i.e.\ with probability $\gamma$. One should not expect a sharp change at $p_2$, since for
%$p_S>p_2$ the probability of sale is the same for IPV and for ASI houses. The reason that
%$1-F_{(1)}(X_B(p_S)))=1-F_{(1)}(p_S)$ is that for $p_S>p_2$, one has
%$X_S(p_S)=p_S$. 
This is illustrated in
Figure~\ref{fig:theory-probability-of-sale}.  



In reality, one would expect that there are more than two
types of properties, that there are different degrees of asymmetric information and hence different lower bounds $\ul p_S$ for the gaps $(\ul p_S,v_J)$. This will smooth out the discontinuity at $\ul p_S$ in
Figure~\ref{fig:theory-probability-of-sale}, but we should still
expect that the probability of sale to increase in the bank's
maximum bid in an interval below $v_J$, and jump downwards at $v_J$. Note that the upper bound of the gap $v_J$ is the same irrespective of the degree of asymmetric information, hence the discontinuity at $v_J$ will not be smoothed out.
We thus arrive at the following testable hypothesis concerning the probability of sale to the broker, denoted as $\rho(p_S)$. We assume that the prices are normalized by $v_J$, so that all houses have the same effective $v_J$.
\begin{figure} %[th!]
\centering
\includegraphics[scale = 0.5]{graphics/prob_of_sale}
\caption{Probability of sale to the broker as a function of the bank's
  maximum bid for independent symmetric information houses
  (dotted, blue), adverse selection houses (red, dashed), and a
  mixture of both types of houses (black, solid).}
\label{fig:theory-probability-of-sale}
%\bigskip
\end{figure}
\begin{figure} %[b!]
\centering
\includegraphics[scale=0.6]{graphics/simulated_simple_discontinuity}
\caption{Simulated example for the probability of sale to the broker. Linear payoffs and lognormal distributions for $x_B,x_S$. 50,000 random draws in a Monte Carlo simulation. Probability of sale as a function of the bank's maximum bid. Locally linear kernel regression with the data split at $p/v_J=1$ (confidence interval: 95\%, Epanechnikov kernel and rule-of-thumb (ROT) bandwidth selection, see \cite{fan1996local}).}
%		
\label{fig:theory-probability-of-sale_sim}
\end{figure}
\begin{conjecture}[Probability of sale]\label{hyp:discontinuity}
If all auctions in the data are under symmetric information, the probability of sale $\rho(p_S)$ is a continuous, decreasing function of $p_S$. If, on the other hand, the data exhibit a mixture of auctions with a varying degree of informational asymmetry, including SI auctions with a positive probability, then we should expect the probability of sale to the broker $\rho(p_S)$ to exhibit the following pattern. Initially, $\rho(p_S)$ decreases in $p_S$. Then, over a certain interval $p_S \in [\tilde p, v_J)$, where $\tilde p < v_J$, $\rho(p_S)$ increases in $p_S$. At $p_S = v_J$, $\rho(p_S)$ drops discontinuously to a lower value, $\rho(v_J)<\lim_{p_S \uparrow v_J} \rho(p_S)$, and from that point on, decreases in a continuous fashion.
\end{conjecture}
This pattern is illustrated in a simulated example in Figure \ref{fig:theory-probability-of-sale_sim}. In this example, it is assumed that the broker's payoff is linear in the signals, $u_B(x_B,x_S) = (1-\beta) x_B+\beta x_S$, and the signals are lognormally distributed. The weight $\beta$ put on the bank's information $x_S$ is uniformly distributed on $[0,1/2]$, with $\beta = 0$ corresponding to symmetric information.



\section{Data}
\label{sec:data}

The Clerk \& Comptroller's auction website provides service for sales on
the foreclosed properties in Palm Beach county, Florida, US. The
website provides a platform for the banks (plaintiff, to whom property
owners hold liability) and potential buyers (mosty brokers), to
meet in this marketplace. The ClerkAuction online platform conducts
foreclosure sales on all business days, which provides a large amount of data on 
these sales.

We collected data from the website for foreclosure sales between
January 21, 2010
and November 27, 2013. Our data record all transaction details on these
sales, including winning bid, winner identities, and judgment
amounts.% and more importantly, the property address in the case.

% With the help of property addresses, we then resort to another database
% powered by Property Appraiser's Public Access in Palm Beach
% county. There, we collected the information regarding the properties
% being sold at the foreclosure sales. In particular, we found the
% appraisal values of property for the current year, type of property,
% and transaction value for next sale after the foreclosure sales.

% We restrict our attention with the auctions
% that had information on all the winner identity, appraisal values, and
% next resale values.  In total, our data set effectively includes 2047
% observations.  

Our dataset contains 12,788 auctions with a total judgment amount
of \$4.2bn. The sum of winning bids is \$0.5bn.
Table \ref{sumstat} reports the summary statistics for
main variables.  The variable \textit{bank winning} indicates
that 84\% of auctions under study ended up having properties
transferred to bank's ownership. 

% \begin{itemize}
% \item prob of having next sale, conditional on bank wins: 0.5673
% \item prob of having next sale, conditional on broker wins: 0.1107
% \end{itemize}

\begin{table}[!htbp]
\centering \caption{Summary statistics \label{sumstat}}
\begin{tabular}{lcccc}
&&&&\\
\hline
\multicolumn{1}{c}{\textbf{Variable}}&\textbf{Mean}  & \textbf{Std. Dev.} 
& \textbf{Min}& \textbf{Max} \tabularnewline
\hline
% bank winning &0.83	&0.38	&0	&1\\
% number of realtors& 2.79& 1.33& 0&11\\
bank wins & 0.813 & 0.390 & 0 & 1 \\
number of third-party bidders & 1.065 & 1.362 & 0 & 14 \\


&&&&\\
\hline
\multicolumn{1}{c}{\textbf{Variable}}&\textbf{Mean}  & \textbf{Std. Dev.} 
& \textbf{1st Percentile}& \textbf{99th Percentile} \tabularnewline
\hline
\multicolumn{3}{l}{\textit{Variables with original scales}}&&\\
% winning bid & 91868.10&	100950.02	&2300.00	&1300100.00\\
% appraisal value & 121020.77&	138637.27&	8756.00	&2200466.00\\
% next resale value &117989.53	&139817.89&	0.00&	2100000.00\\
% judgment amount &317266.69&	580580.07	&4375.63	&20339327.06\\

bank's bid & 210,996 & 1,294,181 & 4,200 & 1,476,989 \\
judgment amount & 329,951 & 1,953,172 & 4,985 & 2,380,127 \\

&&&&\\
\multicolumn{3}{l}{\textit{Variables normalized by judgment amount}}&&\\
% winning bid&
% 0.720	&0.218	&0.202	&1.494\\
% next resale value&
% 0.979 &	0.494&	0.000	&8.796\\
% judgment amount&
% 3.746	&19.485	&0.084&	874.434\\

bank's bid & 0.766 & 1.363 & 0.096 & 1.167 \\
\hline
\end{tabular}
\end{table}

We now turn to empirical tests of our hypotheses. 

Hypothesis~\ref{hyp:bunching} can be easily checked by plotting the cumulative distribution function of banks' bids. Figure~\ref{fig:public-reserve} shows the distribution of banks' maximum bids. Bunching shows up very clearly: there are roughly 4,000 observations bunched at the judgment amount.

%Figure~\ref{fig:bunching} shows a graph with the empirical cumulative distribution of
%bids normalized by the judgment amount. Observe the similarity in
%shape of the curves in Figures~\ref{fig:model_equilibrium} and
%\ref{fig:bunching}. In both there is bunching of the bank's bids at
%the judgment amount. Note that for the numerical example in
%Figure~\ref{fig:model_equilibrium} we assumed that the bank's and the
%realtor's distributions of valuations are the same, hence the overlap
%for bids above the judgment amount. However, this does not necessarily
%have to hold; indeed Figure~\ref{fig:bunching} suggests that this is
%not the case empirically.

%\begin{figure}[htbp]
%	\centering
%        \includegraphics[width=0.6 \textwidth]{graphics/distr-wbid-secret-1broker}%{bunching_judgement}
%	\caption{Distribution of the winning bid in auctions with a secret reserve conditional on the bank losing the auction and bidding against one other bidder (normalized by judgment amounts). Number of observations: 478.}
%        \label{fig:bunching}
%\end{figure}

\begin{figure}[htbp]
	\centering
	\includegraphics[width=0.6 \textwidth]{graphics/distr-maxbid}%{bunching_judgement}
	\caption{Distribution of the bank's maximum bids (normalized by the judgment amounts). Number of observations: 12,788.}
	\label{fig:public-reserve}
\end{figure}


Next, we plot the probability of sale as a function of the bank's
maximum bid (Figure~\ref{fig:discontinuity-data}). Figure~\ref{fig:discontinuity-data} shows the same kernel regression as the one shown for the Monte Carlo simulation in Figure~\ref{fig:theory-probability-of-sale_sim}: it is a locally linear kernel regression, the sample being split into observations below ($p_S<v_J$) and weakly above ($p_S\geq v_J$) the judgment amount. The most
striking feature of the graph is that the probability of sale
increases with the bank's bid just before the judgment amount and drops down discontinuously at $v_J$. 
%Figure \ref{fig:tests_nonsec} shows the estimated probability of sale over a band around the judgment amount, chosen as $[0,9,1.1]$, and over a band below the judgment amount, $[0.7,0.9]$. These bands are chosen sufficiently wide so that the confidence intervals around the estimates, also shown in the figure, are informative. It can be seen that the estimated probability of sale over the lower band ($0.093$) is smaller than the probability over the surrounding band ($0.159$), with the difference $0.047$. Moreover, a formal statistical test reveals that this difference is statistically significant at $5\%$ level: the t-statistic of the difference in means test is $2.77$.


This is exactly what theory predicts in
case that there is heterogeneity in terms of adverse selection (see
Figures~\ref{fig:theory-probability-of-sale} and \ref{fig:theory-probability-of-sale_sim}).

%\begin{figure}[htbp]
%	\centering
%        \includegraphics[width=0.6 \textwidth]{graphics/running_smoother}
%	\caption{Probability of sale to a realtor as a function
%          of the bank's maximum drop-out price (i.e. the bank's reserve price) (symmetric nearest neighbor smoothing). Dots denote individual
%          observations (at the top if the house goes to a realtor, at
%          the bottom if the bank keeps the house).}
%        \label{fig:probability-of-sale}
%\end{figure}

%\begin{figure}[htbp]
%	\centering
%        \includegraphics[scale=0.5]{tests_nonsec}
%	\caption{Probability of sale to a realtor  for \emph{nonsecuritized} mortgages with the bank's reserve below and around the judgment amount, with confidence intervals at $95\%$ level.}
%        \label{fig:tests_nonsec}
%\end{figure}


\begin{figure}
	\begin{center}
		\includegraphics[width=0.6 \textwidth]{graphics/simple_discontinuity}
		\caption{Probability of sale as a function of the bank's maximum bid. Locally linear kernel regression with the data split at $p/v_J=1$ (confidence interval: 95\%, Epanechnikov kernel and rule-of-thumb (ROT) bandwidth selection, see \cite{fan1996local}).\label{fig:discontinuity-data}}
	\end{center}
\end{figure}

%\begin{figure}
%	\begin{center}
%		\includegraphics[width=0.6 \textwidth]{graphics/simulated_simple_discontinuity}
%		\caption{Estimation based on data generated by a Monte Carlo simulation, using the same estimation procedure as shown in Figure~\ref{fig:discontinuity-data}. Probability of sale as a function of the bank's reserve price. Locally linear kernel regression with the data split at $p/v_J=1$ (confidence interval: 95\%, Epanechnikov kernel and rule-of-thumb (ROT) bandwidth selection, see \cite{fan1996local}).\label{fig:discontinuity-simulation}}
%	\end{center}
%\end{figure}

\section{Applications}
\label{sec:applications}

Our theory describes the effects of asymmetric information of the
seller (the bank) on the outcome of the foreclosure auction, taking
into account the judgment amount. Asymmetric information has been
recognized to be highly relevant for the understanding of how markets
works, on policy implications, and on welfare analyses at least since
Akerlof's seminal contribution on the lemon's market \citep{akerlof1970market}. Dealing with all the implications of asymmetric
information on the seller's side in (foreclosure) auctions is beyond
the scope of this article. However, we will describe two examples in
which the presence or absence of asymmetric information plays a
crucial role: securitization of mortgages and the question of judicial
versus non-judicial foreclosures.

\subsection{Securitization}
\label{sec:securitization}

\subsubsection{Overview}

Securitization played an important role during the financial crisis: mortgagees of a large number of securitized mortgages defaulted

test change 3
test change
---
test change 2

There is an ongoing controversy about the securitization of mortgages
in the U.S. Often, the claim is made that
securitization was one of the main causes of the financial
crisis. According to this view, the combination of securitization 
and asymmetric information led to banks being too
lax when granting mortgages, knowing that holders of securitized
assets and the Government Sponsored Enterprises Freddie Mac and Fannie
Mae would ultimately pay the bill. An opposing view is that a lack of
securitization caused the crisis: during the crisis, the issuing of
securitized assets was drastically reduced, leading to less liquidity
for banks, which forced banks to cut back on lending and hence
exacerbating the crisis.\footnote{The latter position is promoted e.g. by \cite{albertazzi2011securitization}. The former position is taken by all of the other papers we cite in the following.}

In the following, we will provide a short description of the
securitization process, focusing on aspects that are relevant for our analysis. We will discuss some of these issues
at the end of this section. For more details we refer the reader
to \cite{keys2008did}, \cite{tirole2011illiquidity},
\cite{brunnermeier2009deciphering}.
% [ADD REFERENCES TO SOME OF THE RESEARCH
% PAPERS AND SOME QUOTES FROM POLITICIANS HERE]

%Figure~\ref{fig:securitization-over-time} illustrates that there was a
%major decrease in the issuance of mortgage-related securities during
%the crisis. Of course, it does not tell whether this decrease was
%because the mortgages granted were of low quality and the market
%eventually found out, or because liquidity in financial markets was
%excessively reduced. As a matter of fact, this is very hard to
%distinguish empirically.
%
%\begin{figure}[htbp]
%	\centering
%        \includegraphics[width=0.8\textwidth]{graphics/securitization-over-time}
%	\caption{Change of the issuance of mortgage-related
%          securities. [Source: Gorton and Metrick (2012)]\label{fig:securitization-over-time}}
%\end{figure}
%

The securitization involves the following steps. First, the
originating bank grants a mortgage to the home owner. Second, in order to get
liquidity, the originating bank sells the cash flows from a pool of
mortgages to a securitization agency, typically one of the Government
Sponsored Enterprises Freddie Mac or Fannie Mae. The securitization
agency splits the pool of assets into tranches and sells the tranches
to investors on the capital market. Third, if a mortgage defaults, then the trustee of the pool of mortgages will apply for a foreclosure at the respective court. Subsequently, the trustee of the mortgage pool will act as the plaintiff in the foreclosure auction.

For a better understanding of the effects of securitization, first
consider a bank's decision whether to grant a mortgage if there is no
possibility to securitize mortgages. For a given interest rate, the
bank is willing to grant a mortgage if the expected loss is
below the spread between the mortgage interest rate and the bank's refinancing interest rate. The expected loss is the product of the probability
of default (PD) and the loss given default (LGD). The loss given
default is the difference between the mortgage value and the
proceedings from selling the property plus administrative costs. The
combinations of PD and LGD that lead to the bank granting a mortgage
are schematically shown in Figure~\ref{fig:mortgages-without-securitization}.

\begin{figure}
	\begin{center}
		\begin{subfigure}[b]{0.5\textwidth}
			\centering
		\includegraphics[width=\textwidth]{graphics/mortgages-without-securitization-grey}
		\caption{Securitization not possible\label{fig:mortgages-without-securitization}}
\end{subfigure}%
~ %
\begin{subfigure}[b]{0.5\textwidth}
		\centering
		\includegraphics[width=\textwidth]{graphics/mortgages-with-securitization-grey}
		\caption{Securitization possible\label{fig:mortgages-with-securitization}}
\end{subfigure}
\caption{Choice of a bank to grant a mortgage based on probability of default and loss given default. The bank is willing to grant a mortgage in the shaded area in which the product of LGD and PD is below a threshold. If securitization is possible, the bank will also grant mortgages that fulfill formal requirements for securitization (hatched area), besides the non-securitized mortgages it keeps on its books (shaded area).}
	\end{center}
\end{figure}
A bank's possibility to securitize mortgages can lead to moral hazard
and adverse selection. We will restrict our discussion to moral
hazard rather than adverse selection, since the existing literature
suggests that moral hazard is the dominant effect (see \cite{keys2008did}).

The moral hazard problem is due to the bank not having an incentive to
put effort into screening applications for mortgages. It costs time
and money to get reliable information about a borrower's probability
of default and about the loss given default.
Securitization agencies
are aware of this moral hazard and choose two countermeasures. First,
only mortgages can be securitized for which the probability of default
is below a threshold, the probability of default being measured in
terms of FICO scores.\footnote{This is a somewhat simplified account,
	since securitization agencies chose two cutoffs for FICO scores:
	above a FICO score of 620, banks can securitize \emph{low
		documentation mortgages}, below a score of 600, it is very
	difficult to securitize mortgages. Between 600 and 620, banks can
	securitize mortgages, but have to provide full documentation.}
Second, the bank has to choose a pool of mortgages and the agency
randomly picks mortgages which would get securitized in order to reduce adverse selection.

However, there is a reason to believe that these measure did not
perfectly solve the problem of adverse selection due to asymmetric information. The literature
so far has focused mainly on FICO scores not being a perfect measure for
the probability of default (see \cite{keys2008did}). However, even
if the FICO score were a perfect measure of PD, uncertainty with
respect to the LGD leads to moral hazard.  As illustrated in Fig.~\ref{fig:mortgages-with-securitization}, the bank has an incentive not to exert sufficiently high
effort in screening borrowers with respect to PD and LGD. In the
stylized extreme case depicted in Fig.~\ref{fig:mortgages-with-securitization}, the bank exerts no
effort in screening borrowers whose FICO score is above a
threshold.

This has two effects. First, mortgages which would not have been
granted without securitization because of a high LGD (relative to the
PD), will be granted with securitization. Second, since the bank
gathers less information about mortgagees, the bank has less private
information about the LGD. The second effect is increased by the fact
that the little soft information gathered by the bank may be lost when
the mortgage is securitized and sold on the capital market.

For the foreclosure auction, this implies that for securitized mortgages, we should expect the bank to be less informated about the quality of the property than for non-securitized mortgages.

%To see how our results on foreclosure auctions fit into the bigger
%picture, consider the following three stage game.
%\begin{enumerate}
%	\item Potential mortgagees apply for a mortgage. The bank observes the
%	FICO score of the mortgagee and chooses whether to collect
%	additional information about the mortgagee and the house for which
%	the mortgage is to be granted. Additional information allows the
%	bank to assess the loss-given-default.
%	\item Mortgages with a sufficiently low probability of default
%	(measured by the FICO score) get securitized and become part of a
%	mortgage pool. All other mortgages remain with the originating bank.
%	\item In case of default,  the lender files a
%	complaint with the court. For non-securitized mortgages, the
%	lender is the local originating bank. For securitized mortgages, the
%	lender is typically represented by the trustee of the mortgage fund,
%	typically a non-local bank. Subsequently, the court runs a
%	foreclosure auction in which the lender and third-party bidders
%	bid. Proceedings from the auction, up to the judgment amount are
%	paid to the lender.
%\end{enumerate}
%
%The focus of our analysis in stage 3. The key point of stages 1 and 2 that is relevant to our analysis is that
%the lender should be better informed about the quality of the property
%for non-securitized than for
%securitized mortgages. One reason is that the bank is less likely to collect information about the loss-given-default at the
%first stage. Another reason is that the information collected may be
%lost at the securitization stage.

This leads us to the empirically testable implication that is an adaptation of Hypothesis~\ref{hyp:discontinuity}. Given that holders of securitized mortgages are less informed about the quality of the house and holders of non-securitized mortgages are better informed, we get the following hypothesis:

\begin{conjecture}[Probability of sale for securitized and non-securitized mortgages]\label{hyp:discontinuity-sec-nonsec}
	For securitized mortgages, the probability of sale $\rho(p_S)$ is a continuous, decreasing function of $p_S$. For non-securitized mortgges, we should expect the probability of sale to the broker $\rho(p_S)$ to exhibit the following pattern. Initially, $\rho(p_S)$ decreases in $p_S$. Then, over a certain interval $p_S \in [\tilde p, v_J)$, where $\tilde p < v_J$, $\rho(p_S)$ increases in $p_S$. At $p_S = v_J$, $\rho(p_S)$ drops discontinuously to a lower value, $\rho(v_J)<\lim_{p_S \uparrow v_J} \rho(p_S)$, and from that point on, decreases in a continuous fashion.
\end{conjecture}

In this description of securitization, we abstracted away from a
number of issues which are orthogonal to our analysis, but matter for
other purposes. Some of these issues are that information may not be
lost, but merely reduced by securitization. The FICO score, which is
used as a threshold for securitization, is an imperfect measure for
the probability of default. Evidence for this is provided by \cite{keys2008did}, who show that mortgages with a FICO score above the
threshold were more likely to be securitized and also had higher
probabilities of default than mortgages with FICO scores slightly
below the threshold. Crossing the FICO score threshold from below
increases the probability of default from 5\% to
5.5\%-6\%.\footnote{To be more precise, there are two thresholds for
	FICO scores. The analysis of \cite{keys2008did} mainly focuses on
	the threshold at a FICO score of 620 below which banks have to
	provide full documentation and below which banks can grant low
	documentation mortgages.} This reinforces our point that the lender
has better information about the loss-given-default for securitized
than for non-securitized mortgages, since the probability of default
is known to be correlated with the loss-given-default \cite{qi2009loss}.

Further, banks are required to keep part
of the securitized mortgages on their balance sheets, so that banks
have an incentive to exert less effort to collect information, not
necessarily no effort at all. This will lead to a noisier, but not to
completely uninformative signal for the lender in case of securitization. However, even with these additional issues we should expect the same prediction: that for a securitized mortgages the plaintiff is less informed than for non-securitized mortgages.

The hypothesis that the bank's information concerning the common value component is \emph{less precise} for the securitized mortgages can be directly tested. We do so assuming the linear model \[u_B(x_B,x_S) = x_B+\alpha x_S .\] Proposition \ref{prop:slope} shows that the bank's strategy $p_S(x_S;\alpha)$ is increasing in $\alpha$. 
If some information is lost in the process of securitization, and the bank's information is therefore less precise for the securitized mortgages, then $\alpha$ is expected to be smaller. This leads to the following testable hypothesis.
%


\begin{conjecture}\label{hyp:slope}
	For bids below $v_J$, we have \[ p_S^{nonsec}(x_S) > p_S^{sec}(x_S), \quad\quad x_S>0 .\] \end{conjecture}

In the next section, we test these hypotheses with our data.

%This is illustrated in
%Figure~\ref{fig:securitization}.
%
%\begin{figure}[htbp]
%	\centering
%       \includegraphics[width=0.8\textwidth]{graphics/securitization2}
%	\caption{Securitization. [Source: Gorton and Metrick
%          (2012)]\label{fig:securitization}}
%  \end{figure}

\subsubsection{Data}

Since we have the name of the plaintiff in each foreclosure auction,
we can categorize mortgages as securitized vs non-securitized. We use
a simple classification rule, we classify a mortgage as securitized if
the name of the plaintiff contains at least one of the following
keywords: "trust", "asset backed", "asset-backed",
"certificate", "security", "securities", "holder".\footnote{Two examples of plaintiff's names that are classified as securitized are "US BANK NATIONAL ASSOCIATION AS TRUSTEE ON BEHALF OF THE HOLDERS OF THE ASSET BACKED SECURITIES CORPORATION HOME EQUITY LOAN TRUST SERIES AEG 2006-HE1 ASSET BACKED PASS-THROUGH CERTIFICATES SERIES AEG 2006-HE1
	HSBC MORTGAGE SERVCIES INC" and "DEUTSCHE BANK NATIONAL TRUST COMPANY AS TRUSTEE FOR ARGENT SECURITIES INC ASSET-BACKED PASS-THROUGH CERTIFICATES SERIES 2006-W2".} This simple
categorization does give false negatives (for some securitized
mortgages none of the keywords shows up in the name of the plaintiff),
but almost no false positives.

We have classified 3,249 mortgages as securitized and 9,539 as non-securitized. Figure~\ref{fig:distr-maxbid-sec-nonsec} plots the distributions of banks' bids for securitized and non-securitized mortgages. The average difference of the bank's bid as a fraction of the judgment amount is much lower for securitized than for non-securitized mortgages, roughly 40 percentage points. This could be explained by two effects. First, a selection
effect that leads to securitized mortgages being backed by houses of
lower quality. Such a selection effect at the foreclosure auction
stage can be due to moral hazard at a previous stage, when mortgages
are granted. Second, the signaling premium leads to banks bidding
higher for non-securitized than for securitized mortgages, since the
common value component plays a larger role.

\begin{figure}
	\begin{center}
		\includegraphics[width=0.7\textwidth]{graphics/distr-maxbid-sec-nonsec}
		\caption{Distribution of the bank's maximum bid as a fraction of the judgment amount $p_S/v_J$ for securitized (dashed line) and non-securitized (solid line) mortgages.\label{fig:distr-maxbid-sec-nonsec}}
	\end{center}
\end{figure}


%\begin{table}[!htbp]
%	\centering \caption{Descriptive statistics for securitized and
%		non-securitized mortgages.[REPORT PUBLIC RESERVE PRICES?]\label{tab:descriptive-securization}}
%	\begin{tabular}{lccc}
%		\hline\hline
%		& Proba. Sale & Sale Price/Judgment Amount Mean %& 25\% Quantile &
%		%50\% & 75\% 
%		& \# \\ \hline
%		All auctions & 16.5\% & 0.689 %& 0.242 & 0.387 & 0.953 
%		& 12,788 \\
%		Securitized & 16.8\% & 0.413 %& 0.212 & 0.307 & 0.416 
%		& 14,087 \\
%		Non-securitized: & 16.4\% & 0.827 %& 0.268 & 0.499 & 1.017 
%		& 28,928 \\
%		%  & Price & Judgment & $\frac{\text{Price}}{\text{Judgement
%		%      value}}$  & \# \\
%		%  & & value & & 
%		% \\ \hline
%		% % All auctions &  76997 & 240964  & 0.358  & 469 \\
%		% %              &  (96312) & (221231) & (0.316) &  \\
%		% All auctions & 76,150 & 308,740 & 0.313 & 12,788 \\
%		%  & (181,014) & (1,308,201) & (1.532) & \\
%		% \hline
%		% % Securitized &  81587 & 277828 & 0.264 & 167 \\
%		% %             &  (93233) & (215225) & (0.181) &  \\
%		% % Non-securitized &  74459 & 220579 & 0.411 & 302 \\
%		% %                 &  (98033) & (222217) & (0.360) &  \\
%		% Securitized & 87,256 & 344,098 & 0.277 & 14,087 \\
%		%  & (119,882) & (990,335) & (2.505) & \\
%		% Non-securitized & 70,742 & 291,522 & 0.330 & 28,928 \\
%		%  & (204,047) & (1,437,471) & (0.658) & \\
%		% % Securitized, bank wins &  76589 (97714) & 273944 (231771) & 0.240 (0.153) & 129 \\
%		% % Securitized, broker wins &  98553 (74732) & 291015 (147505) & 0.345 (0.241) & 38 \\
%		% % Non-securitized, broker wins &  76958 (102873) & 239776 (232476) & 0.311 (0.221) & 231 \\
%		% % Non-securitized, broker wins &  66331 (80405) & 158121 (172004) & 0.736 (0.508) & 71 \\
%		\hline
%	\end{tabular}
%\end{table}

One cannot directly disentangle these two effects, hence it cannot be
seen immediately whether there is more asymmetric information for
non-securitized than for securitized mortgages. However, we can use
our theory as a tool to uncover evidence of asymmetric information. We
will take two approaches for this. The first relies on our theoretical
results on the discontinuity of the probability of sale at the judgment amount. For the second
approach, we use additional data that we hand collected for a subset
of the data set.

\subsubsection{Discontinuity}


Our theoretical results on the judgment amount and the effect of the
common value component help us gain more insight on asymmetric
information. Recall that in the presence of asymmetric information, we
should expect an increase and then a discontinuity in the probability of sale as a function
of the bank's bid (Hypothesis~\ref{hyp:discontinuity-sec-nonsec}).

Figures~\ref{fig:probability-of-sale-securitized} and
\ref{fig:probability-of-sale-nonsecuritized} show the probability of
sale as a function of the bank's bid for securitized
and non-securitized mortgages. For securitized mortgages, there is no increase in the probability of sale below the judgment amount. Further, the discontinuity at $v_J$ is much less pronounced and indeed not statistically significant. For non-securitized mortgages, we still see the same pattern as in Figure~\ref{fig:discontinuity-data}. This is consistent
with the theory that asymmetric information plays less of a role for
securitized mortgages, since the trustee of the mortgage pool is less
likely to have an informational advantage over brokers than a local
bank.

\begin{figure}[tbp]
	%	\centering
	%		\includegraphics[width=\textwidth]{graphics/probability_of_sale_securitized}
	%	\caption{Probability of sale (which is equivalent to the
	%          probability that the bank loses the auction) as a function
	%          of the bank's public maximum bid for securitized
	%          mortgages (LOWESS curve with 10\%
	%          span). Only observations are taken, in which the bank's
	%          maximum bid is public (bank wins in 2,671 auctions, broker
	%          wins in 577 auctions). Circles denote individual
	%          observations (at the
	%          top if the house goes to a realtor, at the bottom if the
	%          bank keeps the house).}
	%        
	\centering
	\includegraphics[width=0.6 \textwidth]{graphics/discontinuity_securitized}
	\caption{Probability of sale as a function of the bank's maximum bid for securitized mortgages. Locally linear kernel regression with the data split at $p/v_J=1$ (confidence interval: 95\%, Epanechnikov kernel and rule-of-thumb (ROT) bandwidth selection, see \cite{fan1996local}).}
	\label{fig:probability-of-sale-securitized}
\end{figure}

\begin{figure}[tbp]
	\centering
	\includegraphics[width=0.6 \textwidth]{graphics/discontinuity_nonsecuritized}
	\caption{Probability of sale as a function of the bank's maximum bid for non-securitized mortgages. Locally linear kernel regression with the data split at $p/v_J=1$ (confidence interval: 95\%, Epanechnikov kernel and rule-of-thumb (ROT) bandwidth selection, see \cite{fan1996local})}
	\label{fig:probability-of-sale-nonsecuritized}        
\end{figure}


%\begin{figure}[htbp]
%	\centering
%        \includegraphics[scale=0.5]{tests_sec}
%	\caption{Probability of sale for \emph{securitized} mortgages below and around the judgment amount, with confidence intervals at $95\%$ level.}
%        \label{fig:tests_sec}
%\end{figure}

\subsubsection{Bidding strategy}

Figures~\ref{fig:probability-of-sale-securitized} and
\ref{fig:probability-of-sale-nonsecuritized} provide indirect evidence
for a common value component for non-securitized mortgages, while there is no such evidence for the securitized mortgages. However, for non-securitized mortgages there are relatively few observations above the judgement value, which might also explain the visual absence of the discontinuity. In this section, we develop and implement a direct test of our Hypothesis \ref{hyp:slope},
\[ p_S^{nonsec}(x_S) > p_S^{sec}(x_S).\]

While the theoretical prediction are clear, the empirical strategy is more involved. It is not sufficient to observe that
banks' bids are higher for non-securitized than for securitized mortgages as depicted in Fig.~\ref{fig:distr-maxbid-sec-nonsec}. The reason for a different distribution of bids may
be either the signaling premium (higher $p_S(x_S)$ for a given $x_S$), a selection effect (different distributions
of $x_S$ for securitized and non-securitized mortgages), or a combination of both. We have to control for $x_S$ in order to
identify the signaling premium.


In order to be able to control for $x_S$, we have hand-collected additional information for a subsample of our data set, namely for properties foreclosed in September 2011. In particular, we collected the information on property transactions after foreclosure sales from a different data source, a website powered by the Palm Beach County Property Appraiser (a government agent). The information regarding the property transactions are for ad valorem tax assessment purposes, as stated in the disclaimer of the website. This implies the appraiser exercises auditing procedures strictly to ensure the validity of any transaction received and posted by that office.

For each foreclosure case, we first traced back the original legal documents, from which we found the property address. We then used the address to search in the database for the detailed features regarding the property and its transaction history. The property appraiser database provides information regarding the type of property (single family, townhouse, zero lot line, etc.), its appraisal values for the most recent three years, the next sale date, the next sale value, and information on the next owner. Using this information, we were able to recover some of the next resale values of the properties at foreclosure.

A comparison of the data on foreclosed properties and the data from the Property Appraiser's database revealed that a perfect matching of observations in the two data sets is not possible, since the address listed in the legal documents is not always of the same format (or containing the same details) as the information recorded in the appraisal database. In order to err on the side of caution, we only used observations for which we can be certain about the identity of the buyer (being the property at foreclosure under study).\AN{Steven, I don't understand what "(being the property at foreclosure under study)" means. Could you please clarify.} 

We identified 332 properties among 644 foreclosed cases in the month.\footnote{There were 983 properties listed for foreclosure for September 2011, but 339 of them were canceled prior to the auction dates.} The next sale of the foreclosed properties happened mostly in the year of 2012, though a few of the properties were not resold until early 2013. This leaves us with 250 observations for which the resale price is available. Since our aim is to get an estimate of the bank's valuation distribution, we only used the data from the foreclosure auctions in which the bank won. This restricts our sample to 199 observations, with 77 of them in the category of securitized properties and 122 cases of non-securitized properties. 

Table~\ref{tab:descriptive-securitization-resale} provides descriptive
statistics of the data collected.\AN{[POSSIBLY ADD COLUMN WITH ASSESSMENT
VALUE/JUDGMENT AMOUNT.]} The table suggests that part of the difference
in sales prices between securitized and non-securitized mortgages is
explained by a selection effect: the bank's resale price in case it
wins the auction is higher for non-securitized than for securitized
mortgages. In the following, we will disentangle the selection and the
signaling premium effect.

\begin{table}[!htbp]
	\centering \caption{Descriptive statistics for securitized and
		non-securitized mortgages. %\AN{[REPORT MEDIAN SALE PRICE/VJ INCLUDING OUTLIERS,
		%ASSESSMENT VALUE/VJ, QUANTILES, MENTION OUTLIERS AS REASON FOR QUANTILES]}
		\label{tab:descriptive-securitization-resale}}
	\begin{tabular}{lccc}
		\hline\hline
		& Resale Price/Judgment Amount Mean %& 25\% Quantile & 50\% & 75\% 
		& \#\\
		All auctions & 0.391 %& 0.245 & 0.359 & 0.495 
		& 199 \\
		Securitized & 0.349 %& 0.222 & 0.324 & 0.444 
		& 77 \\
		Non-securitized: & 0.417 %& 0.254 & 0.399 & 0.546 
		& 122 \\
		\hline
	\end{tabular}
\end{table}

The bank's bidding behavior is determined by its opportunity cost of
selling $x_S$, which is the expected resale
price. We identify the distribution of the unobservable $x_S$ by specifying the following correlation structure between $x_S$, the observable resale price $r$ and the (also observable) tax assessment value $a$. 
%
%
%
%Hence, $x_S$ is a noisy signal of the resale price $r$, in the
%following we will assume that it there is multiplicative
%noise. Formally,
\begin{align*}
\tilde r & = \tilde x_S   + \epsilon_r, \quad \quad \tilde a = \tilde x_S   + \epsilon_a
\end{align*}
where $\tilde x_S:=\ln x_S-E[\ln x_S]$ is the de-meaned log-opportunity
cost of the bank, and $\tilde r$, $\tilde a$ are the de-meaned
log-resale price and tax assessment, respectively. In this specification, the noise terms $\epsilon_r$ and $\epsilon_S$ are mutually independent, and are also independent of $\tilde x_S$.
%
%and $\epsilon_r$ is an error term. Note that we only
%observe $r$, but not $x_S$. To identify the distribution of $x_S$, we
%use additional information about the tax assessment value of a
%property $a$. $a$ is a noisy signal about the bank's opportunity cost
%$x_S$. Assuming multiplicative noise, we can write
%\[
%\tilde a = \tilde x_S   + \epsilon_a
%\]
%where $\tilde a$ is defined analogously to $\tilde x_S$.
%
%Our identifying assumption for the identification of $x_S$ is that
%$\epsilon_r$ and $\epsilon_a$ are independently distributed. 
%[POSSIBLY
%ADD JUSTIFICATION FOR THIS. AN AVENUE WE COULD POTENTIALLY GO IS THAT
%IF $\epsilon_r$ and $\epsilon_a$ WERE NOT INDEPENDENT, THE BANK COULD
%USE THE ASSESSMENT VALUE TO UPDATE ITS ESTIMATE OF THE RESALE PRICE.]
Following \cite{li1998nonparametric} and \cite{krasnokutskaya2011identification}, the 
underlying distribution of $x_S$ is non-parametrically identifiable 
under this assumption. One could use standard non-parametric deconvolution
techniques based on Fourier transforms to obtain the distribution of $x_S$.

However, because of sample size issues, we take a semi-parametric approach
and parametrically deconvolute the noise in the following way.
We can identify the variance of $\tilde x_S$ by considering the
variance of different combinations of $r$ and $a$. A simple example
for this is the following set of variances and the corresponding
equations:
\begin{align*}
	\text{Var}[\tilde r] & = \sigma_r^2 + \sigma_S^2 \\
	\text{Var}[\tilde a] & = \sigma_a^2 + \sigma_S^2 \\
	\text{Var}\left[\frac{1}{2}\tilde r + \frac{1}{2}\tilde a\right] & =
	\frac{1}{4}\sigma_r^2 + \frac{1}{4}\sigma_a^2+\sigma_S^2
\end{align*}
where $\sigma_S^2$, $\sigma_r^2$, and $\sigma_a^2$ are the variances
of $\tilde x_S$, $\epsilon_r$, and $\epsilon_a$, respectively. The above is a system of three
linear equations with three unknowns $\sigma_i^2$, $i \in  \{  S,r,a \}$, and has a unique
solution.

The empirical variances of $\tilde r$, $\tilde a$, and
$\tilde r/2+\tilde a/2$ will thus lead to a consistent estimate of $\sigma_S$. Further,
we use the empirical mean of $r$ as an estimate for the mean of
$x_S$, which is also consistent. We further make the parametric assumption that $x_S$, $r$, and
$a$ are log-normally distributed. Estimates of the distributions of $x_S$ are reported in Table~\ref{tab:xs}. The difference in means reveals that there is indeed a selection effect: securitized mortgages have a lower resale price to judgment amount ratio $x_S/v_J$ than non-securitized mortgages. In the following, we will show that this selection effect cannot explain all the difference between securitized and non-securitized mortgages. 

\begin{table}
	\begin{center}
		\begin{tabular}{l|cc||cc}
			& $\mu_S$ & $\sigma_S$ & mean & std \\
			\hline
			securitized & -1.17236 & 0.429416 & 0.33954 & 0.152791 \\
			non-securitized & -1.01577 & 0.588605 & 0.430615 & 0.277086
		\end{tabular}
		\caption{Distribution of $x_S$ for securitized and non-securitized mortgages. The estimate is based on a log-normal distribution $x_S/v_J\sim \ln N(\mu_S,\sigma_S)$.\label{tab:xs}}
	\end{center}
\end{table}


Next, observe that the distribution of banks' bids satisfies
$G_S(p_S(x_S))=F_S(x_S)$ if there is homogeneity with respect to the
common value component. Since we only observe the distribution of the
bank's resale prices if the bank wins the auction, it is useful to
define $\tilde F_S(x_S)$ and $\tilde
G_S(p_S)$ the distributions of $x_S$ and $p_S$ conditional on winning the auction.
%and 
%%as
%%$\tilde F_S(x_S):=F_S(x_S)/(1-F_{(1)}(p_S(x_S)))$. Similarly, let 
%$\tilde
%G_S(p_S):= G_S(p_S)(1-F_S(p_S))$. 
Because \[ \tilde F_S(x_S) =\tilde G_S(p_S(x_S)), \]we can write
\[
p_S(x_S) = \tilde G_S^{-1}(\tilde F_S(x_S))
\]
Given that we have estimates of $\tilde G_S$ and $\tilde F_S$ both for
securitized and for non-securitized mortgages, we can estimate $p_S$
for both types of mortgages.

%Note that without unobserved heterogeneity, the function $\tilde G_S^{-1}(\tilde F_S(x_S))$ can be interpreted as the bidding function $p_S(x_S)$. With unobserved heterogeneity it should be interpreted as an "average bidding function". If for some non-securitized mortgages there is a private value auction and for others a common value component auction, then $\tilde G_S^{-1}(\tilde F_S(x_S))$ will be a weighted average of $p_S^*(x_S)$ and $p_S^0(x_S)$. This average would still be higher than the bidding function $p_S^0(x_S)$ for securitized mortgages by Corollary~\ref{cor:bidding-sec-nonsec}. See Appendix~\ref{sec:avg-bid-heterogeneity} for more details.


Figure~\ref{fig:ps-xs-sec-nonsec} shows estimates of $p_S(x_S)$ for securitized and non-securitized mortgages. The estimates are consistent with the first alternative in our Hypothesis \ref{hyp:sp}, namely that the bank's bid is higher for non-securitized than for securitized mortgages, controlling for the bank's opportunity cost $x_S$.

Note that comparing the bidding functions for bids below $v_J$ is difficult, since the bidding function below $v_J$ has a different support for securitized and non-securitized mortgages. It is more practical to compare the inverse bidding function $x_S(p_S)$ for securitized and non-securitized mortgages, since the support is $[0,v_J]$ in both cases. The inverse bidding functions are shown if Figure~\ref{fig:xs-ps-sec-nonsec}. Computing the 90\% confidence interval of the difference of the inverse bidding functions for securitized and non-securitized mortgages reveals that the difference is significant for high values of $p_S$, see Figure~\ref{fig:xs-diff-10}. 
%For low values of $p_S$, the difference is not significant and close to 0. This is consistent with the theoretical prediction that the signaling premium is 0 at the lower bound of support.

\begin{figure}
	\begin{center}
\begin{subfigure}[b]{0.5\textwidth}
		\centering
		\includegraphics[width=\textwidth]{graphics/ps_xs_semiparametric}
		\caption{Bidding function\label{fig:ps-xs-sec-nonsec}}
\end{subfigure}%
~%
\begin{subfigure}[b]{0.5\textwidth}
		\centering
		\includegraphics[width=\textwidth]{graphics/xs_ps_semiparametric}
		\caption{Inverse bidding function\label{fig:xs-ps-sec-nonsec}}
\end{subfigure}
\caption{Bidding function $p_S(x_S)$ and inverse bidding function $x_S(p_S)$ relating the bank's opportunity cost of selling $x_S$ and the bank's bid $p_S$ for securitized (dashed line) and non-securitized (solid line) mortgages. The functions are constructed by matching quantiles of the estimated distribution of $x_S$ with the quantiles of observed bids submitted by banks.}
\end{center}
\end{figure}

\begin{figure}[htp]
	\begin{center}
		\includegraphics[width=0.7\textwidth]{graphics/xs-diff-10-line}
		\caption{Difference between banks' inverse bidding functions (i.e. the bank's opportunity cost of selling $x_S$ as a function of its bid $p_S$) for securitized and non-securitized mortgages. The difference (solid line) is constructed by matching quantiles of the estimated distribution of $x_S$ with quantiles of observed reserve prices submitted by banks. The 5\% and 95\% percentiles of the difference estimate (dashed lines) are constructed using a bootstrap estimate.\label{fig:xs-diff-10}}
	\end{center}
\end{figure}

The above analysis provides evidence for the existence of asymmetric information in mortgage markets. The analysis uncovers a difference between securitized and non-securitized mortgages and is consistent with the hypothesis that there is indeed information about the loss given default lost in the securitization process. This is consistent with the often held suspicion that there is moral hazard in the securitization process: if a mortgage is expected to be securitized, then the originating bank has less incentives to collect information about the quality of the mortgage and will be hence less informed than for non-securitized mortgages. While the literature so far has focused on moral hazard with respect to acquiring information about the probability of default, our analysis has uncovered another dimension along which there is moral hazard: The bank exerts insufficient effort to collect information about the loss given default. Our results suggest that the effect of securitization on the loss given default is similarly important for the expected loss as the effect on the probability of default. The bank's resale price as a function of the judgment amount is 21\% lower for securitized than for non-securitized mortgages (0.340 vs 0.431).\footnote{We provide results for the resale prices rather than losses given default, since there is more reliable data on this. A rough estimate of the difference in terms of the loss given default can be obtained by assuming that the loss given default is $v_J-x_S$. This would mean a 16\% higher loss given default for securitized than for non-securitized mortgages (1-0.340 versus 1-0.431). The loss given default is typically higher than $v_J-x_S$, because of administrative costs.} This is comparable to the 10\% to 20\% increase of the probability of default for securitized versus non-securitized mortgages (5.5\% to 6\% versus 5\%) estimated by \cite{keys2008did}.



%\subsection{Non-Judicial Foreclosures}
%\label{sec:non-judic-forecl}
%
%In this paper we have developed a theory of judicial foreclosures,
%i.e. foreclosures that are organized by a court. Roughly half of the
%states in the U.S. (including Florida) only allow judicial
%foreclosures. The other half of the states allow both judicial and
%non-judicial foreclosures. In a non-judicial foreclosure, the lender
%typically has the \emph{power of sale} and can seize and sell the
%house without going through a court. In the law literature on foreclosure
%reform, it has been argued that allowing the bank to seize the property and actively marketing it would simplify the sales process \cite{nelson2004reforming}. It would also make it easier for realtors to inspect the property, thereby reducing -- even if not completely removing -- asymmetric information. In the following, we show the relevance of different foreclosure procedures. To make the discussion simple, we take the extreme stance that a non-judicial foreclosure allows to completely remove the asymmetric information. In practice, one would expect that asymmetric information would be reduced rather than completely removed.\footnote{In practice, the non-judicial foreclosure procedure is also less extreme than described here. Only Delaware \citep{ghent2011recourse} has the extreme case of strict non-judicial foreclosures in which all proceedings of the sale go to the bank and the original owner gets nothing. Other states with non-judicial foreclosures have limited borrower protection, see \cite{ghent2011recourse}.}
%
%Note that comparing the utility of the
%original owner would be trivial: with non-judicial foreclosures, the
%original owner does not get anything, whereas with a judicial
%foreclosure, he may get something, so he is trivially better off.
%
%Therefore, instead of the original owner's utility, we will compare
%overall welfare in judicial and non-judicial foreclosures. Note that
%we only need to consider the allocation rule, since transfers do not
%matter for welfare. An allocation rule consists of the probability
%$Q_S(\boldsymbol x_B,x_S)$ that the seller gets the house and the
%probability $Q_B^i(\boldsymbol x_B,x_S)$ that buyer $i$ gets the
%house as a function of the seller's signal $x_S$ and the vector of
%buyers' signals $\boldsymbol x_B=(x_B^i)_{i=1}^n$. Welfare is then the
%expected sum of utilities for a given allocation rule:
%\begin{equation}
%	\label{eq:welfare}
%	\int_0^\infty ... \int_0^\infty \left[\sum_{i=1}^n Q_B^i(\boldsymbol
%	x_B,x_S) u_B(x_B^i,x_S)+Q_S(\boldsymbol x_B,x_S) u_S(x_S)\right]
%	dF_B(x_B^1)...dF_B(x_B^n)dF_S(x_S)
%\end{equation}
%Observe that $u_B(x_B,x_S)>u_S(x_S)$ iff $x_B>x_S$ since
%$u_B(x,x)=x$ and $u_B$ and $u_S$ are weakly increasing in their
%arguments. Therefore, the social planner would maximize \eqref{eq:welfare} by giving the
%house to the bidder with the highest signal $x$. Formally, the
%first-best allocation rule is
%\[
%Q_S^*(\boldsymbol x_B,x_S) =
%\begin{cases}
%1 & \text{if $x_S>\max_i x_B^i$,} \\
%0 & \text{otherwise,}
%\end{cases}
%\]
%and
%\[
%Q_B^{i*}(\boldsymbol x_B,x_S) =
%\begin{cases}
%1 & \text{if $Q_S^*(\boldsymbol x_B,x_S)=0$ and $x_B^i>x_B^j$ for all $j\neq i$,} \\
%0 & \text{otherwise.}
%\end{cases}
%\]
%As usual, an arbitrary tie-breaking rule can be specified for the
%zero-probability event that two signals are exactly the same.
%
%For a non-judicial foreclosure auction, we assume that the bank sells
%the house in a standard auction with the same buyers with the same
%valuations showing up as in a judicial foreclosure auction. However, the bank is able to transmit the information to the realtors directly, by advertising the property and allowing inspections. Thus the nonjudicial foreclosures will be described by independent private values model, with $x_S$ directly revealed to the realtors.
%
%\paragraph{Independent Private Values}
%In an independent private values setup, the bank sets the reserve
%price equal to $J_B^{-1}(x_S)$ for all values of $x_S$ in a non-judicial foreclosure where the virtual valuation function is defined as $J_B(x_B):=x_B-(1-F_B(x_B))/f_B(x_B)$. Thus the marginal buyer's cutoff is also equal to $J_B^{-1}(x_S)$. In a judicial foreclosure, the marginal buyer type is given by $J_B^{-1}(x_S)$ for $x_S<J_B(v_J)$ and is equal to $x_S$ for $x_S>v_J$.  
%
%Figure~\ref{fig:judicial-ipv} shows the boundaries of the different
%allocation rules for the one buyer case.  In first-best the buyer gets
%the house, if the realization of signals is above the 45 degree line
%(dashed). For judicial foreclosures, the buyer gets the house above
%the dotted blue line. For non-judicial foreclosure, the buyer gets the
%house above the sold red line. It is straightforward to show that
%welfare is higher with judicial foreclosures than with non-judicial
%foreclosures, since the deadweight loss of monopoly (the area between
%the dashed and solid line for judicial, the area between the dashed
%and the dotted line for non-judicial foreclosures) is smaller.
%
%\begin{figure}[tbp]
%	\centering
%	\includegraphics[width=0.5\textwidth]{graphics/judicial-ipv}
%	\caption{Judicial versus non-judicial foreclosure with independent
%		private values. Lines separating the regions in which the seller
%		and in which the buyer gets the house for first-best allocation
%		(black dashed 45 degree line), for judicial foreclosure (solid
%		red), and non-judicial foreclosure (blue dotted).}
%	\label{fig:judicial-ipv}
%\end{figure}
%
%\paragraph{Common Values}
%With adverse selection, the situation is more complicated as
%illustrated in Figure~\ref{fig:judicial-common-values} for the one
%buyer case. The blue dotted line corresponds to the non-judicial foreclosure and comes from the solution under private values, where the realtor knows $x_S$. The marginal buyer type with whom the $x_S$-type bank will trade is found at the solution to the ``marginal revenue equals cost" equation, \[ J_B(x_B^{nonjud}(x_S),x_S) = x_S.\] The red solid line is the allocation rule for the judicial foreclosure implied by the seller's
%pricing behavior derived in our paper, i.e. $X_B(p_S(x_S))$. Observe
%that, due to the signalling premium, the solid line is above the dotted line for $x_S<x_B^2$. However,  since the judicial auction is fully efficient for $x_S>x_B^2$, the solid line is below the dotted line for $x_S>x_B^2$. Hence, it is ambiguous whether
%welfare would be larger or smaller for non-judicial foreclosures. 
%\begin{figure}[tbp]
%	\centering
%	\includegraphics[width=0.5\textwidth]{graphics/judicial-common-values1}
%	\caption{Judicial versus non-judicial foreclosure with a common
%		value component. Lines separating the regions in which the seller
%		and in which the buyer gets the house for first-best allocation
%		(blue dotted 45 degree line), for judicial foreclosure (solid
%		red), and non-judicial foreclosure (blue dashed).}
%	\label{fig:judicial-common-values}
%\end{figure}
%
%One gets a clear-cut welfare comparison if one lets $v_J$ go to infinity. In this case, the reserve price set in a non-judicial foreclosure auction $p_S^0(x_S)$ is lower than the price set in a judicial foreclosure $p_S^*(x_S)$ due to the signaling premium. Since $p_S^*(x_S) > p_S^0(x_S) > x_S$ and the first-best reserve price is $x_S$, non-judicial foreclosures are more welfare efficient.
%
%While a welfare ranking of judicial versus non-judicial foreclosures depends on market conditions, our results do suggest some welfare comparisons. For instance, for securitized mortgages, judicial foreclosures are likely to be more efficient than non-judicial foreclosures, since the relevance of the common value component seems to be relatively small.
%
%On the other hand, in times of financial distress, when many mortgages are "under water", the price is usually low compared to the judgment amount. Hence, $v_J\rightarrow\infty$ is a reasonable approximation and non-judicial foreclosures are more efficient.
%
\section{Conclusions}

We develop a novel theory of foreclosure auctions and test some of its predictions with data from Palm Beach county (Florida, US). We find evidence for strategic bidding and asymmetric information, with the bank being the informed party. First, the data reveal bunching in bids at the judgement amount as the theory predicts either under symmetric or asymmetric information. Second, there is a discontinuity in the probability of sale to the brokers, as the theory predicts under asymmetric information. Third, by looking separately at securitized and non-securitized mortgages, we find that banks bid lower for the securitized mortgages. Our theory predicts lower bids when the bank is less informed.

Recent literature, e.g.\ \cite{keys2008did} has emphasized the role of soft information as the predictor of the probability of default.  Only hard, verifiable information is priced when the mortgage is securitized. Knowing that the mortgage will be securitized, the lender may not have sufficient incentives to collect soft information. On the other hand, \cite{qi2009loss} present evidence that the probability of default and loss given default are positively correlated. Thus, soft financial information is relevant for the property resale values, and loss of this information for the securitized mortgages means also less information about the resale values, thereby leading to lower bids in the foreclosure auctions.

The implication of our findings in terms of the policy debate about securitization is that there is an additional reason to be cautious about securitization. This speaks in favor of some of the policies that have been proposed, such as requiring better documentation for securitized mortgages, reducing government support for securitization through securitization agencies, or holding the originating bank liable in case a mortgagee defaults (as is the case for European covered bonds), see e.g. \cite{keys2008did}, \cite{tirole2011illiquidity}, and \cite{campbell2013mortgage}.

%
%which implies that our finding is consistent with the moral hazard story.
%
%
%To the extent that such information is correlated with the value of the house, our results are consistent with the moral hazard explanation. 
%
%Our theory has the
%following main empirically testable predictions: (i) that banks' bids are
%bunched at the judgment amount, (ii) if there are both independent
%private values and common value component auctions, there will a non-monotonicity and discontinuity of the probability of sale in the bank's bid
%(the probability of sale increases with the bank's bid just below and drops down at the judgment amount). Using a novel data set, we show that
%predictions (i) and (iii), but not (ii) are consistent with the
%data. This can be interpreted as both independent private values and
%common value component auctions showing up in the data set. Further,
%the data are consistent with the claim that adverse selection plays
%less of a role for securitized than for non-securitized
%mortgages. This is consistent with the idea that local banks with
%non-securitized mortgages have better information about the common
%value component than non-local banks who act as trustees for pools of
%securitized mortgages. These findings speak in support of policy proposals that seek to restrict securitization. 

Our theory can also be used for a welfare comparison of judicial and non-judicial foreclosures. 
%Besides securitization as an application of our theory, we provide a second example of an application: the welfare comparison of judicial and non-judicial foreclosures.
Roughly half of the states in the
U.S. (including Florida) only allow judicial foreclosures, i.e. the
foreclosure auction has to be run by a court with rules as the ones
described in this article. The other half of the states allow for both
judicial and non-judicial foreclosures.  Mortgage contracts
with a so called a ``power of sale'' clause allow the bank to choose a non-judicial
foreclosure in case of a failure to repay, i.e. the bank can directly
seize the property and sell it without going through a court. 

The power of sale clause allows the bank to market the property, and to verifiably disclose, through inspections, the information regarding its condition. This eliminates the signalling premium, but introduces another distortion. The bank, acting as a de facto owner of the property, is no longer obligated to pay the owner back any auction proceed above the judgment amount. Thus the monopoly price distortion now extends to prices above the judgment amount. So the overall welfare effect of the power of sale is ambiguous. We plan to estimate this effect in future work.

\bibliography{foreclosures}
%\bibliography{bilateral}
%\bibliographystyle{econometrica}
%\bibliographystyle{plainnat}
\bibliographystyle{ecta}
\appendix
\section*{Appendix}
\section{Omitted Proofs}
\begin{proof}[Proof of Proposition \ref{prop:nonjudicial}]
For the exposition, we assume $n \geq 2$. The proof for $n=1$ is parallel. It turns out convenient to restate the problem a bit differently. Consider the bank of type $x_S$ that contemplates a dropout price $p$, and assume that the brokers hold the belief $\hat x_S$ concerning the bank's type following the bank's dropout. For now, this belief is not necessarily the equilibrium belief. 
%The broker's  dropout strategy is then given by $u_B(x_B, \hat x_S)$ and the broker's cutoff $\hat X_B(p)$ is determined from the indifference condition \[ u_B(\hat X_B(p),\hat x_S) = p .\]

We restrict attention to equilibria where the broker's dropout strategy (if the bank has not dropped out) is a continuous, strictly increasing function, with a differentiable inverse $X_B^*(p)$. Then the bank's expected profit, as a function of own type $x_S$, the perceived type $\hat x_S$, and the price $p$ is given by 
\begin{align}
  \hat \Pi(x_S,\hat x_S,p) &= p (F_{(2)}(X_B^*(p)) - F_{(1)}(X_B^*(p)))\nonumber\\
  &+\int_{ X_B^*(p)}^{\infty} u_B(x,\hat x_S) f_{(2)}(x)dx 
  +x_S F_{(1)}(X_B^*(p)).
   \label{eq:PiS}
\end{align}
%Here, we provisionally assume that the broker's dropout strategy is a continuous, strictly increasing function, with a differentiable inverse. This implies that the broker types with $x_B > X_B (p)$ will prefer to continue bidding, while the types $x_B < \hat X_B(p)$ will drop out earlier.
A direct calculation shows that
\begin{align} \frac{\partial \hat \Pi_S}{\partial p} &= 
(p-u_B(X_B^*(p),\hat x_S)) f_{(2)}(X_B^*(p))X_B^{*'}(p) \nonumber
\\
&+(x_S - p) f_{(1)}(X_B^*(p))X_B^{*'}(p) \nonumber
\\
&+F_{(2)}(X_B^*(p)) - F_{(1)}(X_B^*(p))
%-n f_B(\hat X_B(p))F_B(\hat X_B(p))^{n-1} \Big(  J_B(\hat X_B(p),\hat x_S)  - x_S \Big) \hat X_B'(p) 
\label{eq:sp}
\end{align}
and 
\begin{align} \frac{\partial \hat \Pi_S}{\partial \hat x_S} = \int_{\hat X_B(p)}^{\infty} \frac{\partial u_B(x,\hat x_S)}{\partial \hat x_S} f_{(2)}(x)dx  .\end{align}


The bank's expected profit
following a deviation from the equilibrium price to some other
price $\hat{p}$ is equal to $\hat \Pi_{S}(x_{S},X_{S}(\hat{p}),\hat{p})$.
In equilibrium, such a deviation should not be profitable,so that the following first-order condition (FOC) must hold for $p\in(\underline{p},\infty)$:
\[
\frac{\partial \hat \Pi_{S}(x_{S},X_{S}(p),p)}{\partial p}=0.
\]
Substituting into this FOC the bank's type that bids $p$, $x_{S}=X_{S}^*(p)$, we obtain the following differential equation: 
\begin{align}
\frac{d X_{S}^*(p)}{dp} & =-\frac{\partial\hat \Pi_{S}(X_{S}^*(p),X_{S}^*(p),p)/\partial p}{\partial\Pi_{S}(X_{S}^*(p),X_{S}^*(p),p)/\partial\hat{x}_{S}}
%\\
%&=\frac{-n f_B(X_B^*(p))F_B(X_B^*(p))^{n-1} \Big(  J_B(X_B^*(p),X_S^*(p))  - X_S(p) \Big)}
%{\int_{\hat X_B(p)}^{\infty} \frac{\partial u_B(x,X_S^*(p))}{\partial \hat x_S} f_{(2)}(x)dx} \frac{dX_B^*(p)}{dp}
\label{eq:ivp}
\end{align}

We now turn to the broker's equilibrium dropout strategy if the bank has not yet dropped out. We claim that the broker's strategy $X_B^*(p)$ defined as the solution (which will be shown unique later in the proof) to \begin{align}u_B(X_B^*(p),X_S^*(p)) = p,\label{eq:ubinproof}\end{align} is  a best response. Consider first the scenario when it is known to the broker that the bank will drop out at price $p$. 
Then it is optimal for the broker to drop out at the price $u_B(x_B,X_S^*(p))$, and the brokers with $x_B < X_B^*(p)$ will drop out at prices lower than $p$, while the brokers with $x_B > X_B^*(p)$ will drop out at higher prices.  
If  $X_B^*(p)$ defined by \eqref{eq:ubinproof} is increasing, it follows that it is optimal to drop out at price $p'$ for a broker of type $X_B^*(p')$. 
Since this best response does not depend on $p$, we see that $X_B^*(p')$ is a best response also when the broker does not know the bank's dropout price $p$. 

Totally differentiating \eqref{eq:ubinproof} yields another differential equation linking $X_S^*(p)$ and $X_B^*(p)$:
 \begin{align} 
% u_B(X_B^*(p),X_S^*(p)) = p \implies
\frac{\partial u_B}{\partial x_B}\frac{d X_B^*(p)}{dp}
+ \frac{\partial u_B}{\partial x_S}\frac{d X_S^*(p)}{dp}= 1.
\label{eq:broker}
\end{align} 
Equations \eqref{eq:ivp} and \eqref{eq:broker} form a linear system for $\frac{dX_S^*(p)}{dp}$ and $\frac{dX_B^*(p)}{dp}$; solving this system yields \eqref{eq:sdifeq} and \eqref{eq:bdifeq} in Proposition \ref{prop:nonjudicial}:  \begin{align}
 \frac{d X_S^*(p)}{dp}&= 
  \frac{(J_B(X_B^*,X_S^*)-X_S^*)]f_{(1)}(X_B^*)}{
  \frac{\partial u_B}{\partial x_S}
   (u_B(X_B^*,X_S^*)-X_S^*)f_{(1)}(X_B^*)+
   \frac{ \partial u_B}{\partial x_B }
    \int_{X_B^*}^\infty \frac{  \partial u_B} {\partial x_S} 
    f_{(2)}(x) dx
    }, \label{eq:sinproof}
    \\ \nonumber
    \\
  \frac{d X_B^*(p)}{dp}&= 
  \frac{  F_{(2)}(X_B^*) -  F_{(1)}(X_B^*) +  \int_{X_B^*}^\infty \frac{  \partial u_B} {\partial x_S} 
    f_{(2)}(x) dx  }{
  \frac{\partial u_B}{\partial x_S}
   (u_B(X_B^*,X_S^*)-X_S^*)f_{(1)}(X_B^*)+
   \frac{ \partial u_B}{\partial x_B }
    \int_{X_B^*}^\infty \frac{  \partial u_B} {\partial x_S} \label{eq:binproof}
    f_{(2)}(x) dx
    } ,
\end{align}
subject to the initial conditions $X_S^*(\underline p) = 0$ and $X_B^*(\underline p) = \underline x$, where the cutoff $\underline x$ is uniquely determined from  $u_B(\underline x, 0) = \underline p$.

With a given value for $\underline x$, the solution to the system \eqref{eq:sinproof} and \eqref{eq:binproof} is unique by standard results in the theory of differential equations. We now claim that the only value of $\underline x$ compatible with equilibrium is the one given in the proposition, namely with $\underline x = \underline x_B$, uniquely determined from $J_B(\underline x_B,0) = 0$. This value corresponds to the full information outcome.

Since $\hat X_B'(p) > 0$, it follows that the expected profit function $\hat \Pi (x_S,\hat x_S,p)$ satisfies the following \emph{single-crossing} condition: the ratio of the slopes
\begin{align}
\frac{\partial\hat\Pi_{S}(x_{S},\hat{x}_{S},p)/\partial p}{\partial\hat \Pi_{S}(x_{S},\hat{x}_{S},p)/\partial\hat{x}_{S}}
\label{eq:sc}
\end{align}
is increasing in $x_{S}$. This single-crossing condition implies that there are no profitable within-equilibrium deviations. Indeed, if  $\hat{p}\geq\underline{p}$ is such a deviation, then differntial equation (\ref{eq:ivp}) implies
\begin{align*}
\frac{dX_{S}^*(\hat{p})}{dp} & =-\frac{\partial\hat \Pi_{S}(X_{S}^*(\hat{p}),X_{S}^*(\hat{p}),\hat{p})/\partial p}{\partial\hat\Pi_{S}(X_{S}^*(\hat{p}),X_{S}^*(p),p)/\partial\hat{x}_{S}}
\end{align*}
This, however, contradicts the single-crossing condition, according to which $X_{S}^*(\hat{p})<x_{S}$
implies 
\[
\frac{\partial\Pi_{S}(X_{S}^*(\hat{p}),X_{S}^*(\hat{p}),\hat{p})/\partial p}{\partial\Pi_{S}(X_{S}^*(\hat{p}),X_{S}^*(\hat{p}),\hat{p})/\partial\hat{x}_{S}}<\frac{\partial\Pi_{S}(x_{S},X_{S}^*(\hat{p}),\hat{p})/\partial p}{\partial\Pi_{S}(x_{S},X_{S}^*(\hat{p}),\hat{p})/\partial\hat{x}_{S}}
\]
Similarly, we can rule out an within-equilibrium deviation to a price
$\hat{p}\in(p,\infty)$.

We now claim that only the full information cutoff $\underline x = \underline x_B$ is compatible with the separating equilibrium. First, observe that any value $\underline x < \underline x_B$ corresponds to the solution of the system that is not monotonically increasing and therefore cannot correspond to a separating equilibrium. Next, if $\underline x > \underline x_B$, then it is profitable for the type $x_S=0$ to deviate to $p < \underline p$ even when the brokers' belief are the most pessimistic, $\hat x_S = 0$. The slope if the expected profit \eqref{eq:sp} is of the same sign as \[ -J_B(\underline x,0)  < J_B(\underline x,0)=0, \text{ if } \underline x > \underline x_B .\]

The cutoff  $\underline x = \underline x_B$ does indeed yield the solution $(X_B^*(p),X_S^*(p))$ in which each function is monotonically increasing in $p$. First, note that equations \eqref{eq:sinproof} and \eqref{eq:binproof} form an autonomous system of differential equations. Every solution curve that is entirely contained in the region \[\mathcal M := \{ (x_B,x_S): x_S\geq 0, J_B(x_B,x_S) - x_S \geq 0   \] corresponds to a monotone increasing solution because the r.h.s. of \eqref{eq:sinproof} and \eqref{eq:binproof} are non-negative in (positive in the interior of) $\mathcal M$. Next, observe that the solution curve that starts at the point $(x_B,x_S) = (\underline x_B,0)$, i.e.\ in the south-west corner of $\mathcal M$ will never leave $\mathcal M$. The left boundary of $\mathcal M$ is given by the full information outcome, $\{ (x_B,x_S):  J_B(x_B,x_S) - x_S = 0 \}$. This boundary is an increasing locus because $J_B(x_B.x_S)$ is assumed increasing in $x_B$. The vector field of the system points inside $\mathcal M$, with \ $dX_S^*(p)/dp = 0$ and $dX_B^*(p)/dp \geq 0$. Therefore, any solution with the initial condition in $\mathcal M$ will stay in $\mathcal M$. Since the initial condition $(\underline x_B, 0) \in \mathcal M$, we conclude that $X_B^*(p)$ and $X_S^*(p)$ are increasing in $p$.
%
%that corresponds to the full information outcome. Our assumption that $J_B(x_B.x_S)$ is increasing in $x_B$ ensures that this boundary is an increasing locus, and the inspection of the r.h.s. of \eqref{eq:sinproof} reveals that the vector flow points inside $\mathcal M$, with \ $dX_S^*(p)/dp = 0$ and $dX_B^*(p)/dp \geq 0$.
%
%
% $\mathcal M \equiv \{ (x_B,x_S): x_S\geq 0, J_B(x_B,x_S) - x_S \geq 0   \}$ where the r.h.s. of \eqref{eq:sinproof} is non-negative (positive in the interior). 


Finally, in order to ensure that the bank with $x_S = 0$ indeed prefers to choose $\underline x_B$, it is sufficient to assume that for any lower (out of equilibrium) price, the brokers believe that the bank's type that deviated is the lowest one, $\hat x_S = 0$. Then \eqref{eq:sp} implies that the slope of the expected profit is \emph{positive} for $p < \underline p$.


\end{proof}

\begin{proof}[Proof of Proposition \ref{prop:slope}]
Instead of the bank's bid strategy directly, in this proof it turns out convenient to work with the function $s(x_B)$, which gives the the bank's type $s$ as a function of the broker's type $x_B$ that would drop out at the same price as the bank. Then the bank's bidding strategy is given by
\begin{align}
p_S(x_S) &= u_B(s^{-\mathbbm 1}(x_S),x_S) \\&=  s^{-\mathbbm 1}(x_S)+\alpha  x_S .
 \end{align}
Dividing equation \eqref{eq:sdifeq} by \eqref{eq:bdifeq}, we get the following differential equation for $s(x_B)$,
\[    
\frac{ds}{dx_B} = \frac{(\bar J(x_B) - (1-\alpha) s)f_{(1)}(x_B)}{\alpha \bar F_{(1)}(x_B)}
\]
where \[ \bar J(t) = t-\frac{1-F_B(t)}{f_B(t)} \] and $\bar F_{(1)}(x_B) = 1- F_{(1)}(x_B)$.

%\cite{cai} consider an equivalent linear model. In their model, \[ u_B(\tilde x_B, \tilde x_S) = \tilde x_B+\tilde x_S, \quad\quad u_S(\tilde x_S) = \gamma \tilde x_S, \quad \gamma > 1.\] Their specification can be seen equivalent to ours if we put \[ \tilde x_B = x_B - \alpha \mathbbm E x_S, \quad\quad \tilde x_S = \alpha x_S, \quad\quad \gamma = \frac{1}{\alpha} .\]In this specification, \[ p_S(\tilde x_S) = m (\tilde x_S) + \tilde x_S, \]
%where $m(\tilde x_S)$ is the the broker's type $\tilde x_B$ that would drop out at the same price as the bank with type $\tilde x_S$. Since $\tilde x_S $ and $x_S$ have the same support $\mathbbm R_+$, this implies that also for $p_S$ as a function of $x_S$, 
%\[ p_S(x_S) = m (x_S) +  x_S. \]
%
%
%
%Theorem 3 in \cite{cai} shows that $s(\cdot) \equiv m^{\mathbbm{-1}}(\cdot)$,  t
This equation can be integrated explicitly\footnote{The integration method is essentially the same as in the proof of Theorem 3 in \cite{cai2007reserve}. However, our linear specification is different and is not a special case of the linear specification in \cite{cai2007reserve}.}
\begin{align*}
s(x_B) = \gamma \bar F_{(1)}(x_B)^{\gamma - 1} \int_{\underline x_B}^{x_B} \bar F_{(1)}(t)^{-\gamma} f_{(1)}(t)\bar J(t)dt
\end{align*}
where \[ \gamma = \frac{1}{\alpha}\] and   $\ul x_B$ is the lowest broker participating type, the same as under symmetric information, \[ \bar J(\ul x_B) = 0.\]
The slope of $s(x_B)$ is given by
\begin{multline}
s'(x_B) =\gamma  (\gamma - 1) f_{(1)}(x_B) \bar F_{(1)}(x_B)^{\gamma - 2} \int_{\underline x_B}^{x_B} \bar F_{(1)}(t)^{-\gamma} f_{(1)}(t)\bar J(t)dt
+\gamma  \bar F_{(1)}(x_B)^{-1} f_{(1)}(x_B) \bar J(x_B) 
\end{multline}
With a change of variable
\[
y = \log \frac{\bar F_{(1)}(x_B)}{\bar F_{(1)}(t)},
\]
we have
\[
s'(x_B) = f_{(1)}(x_B)\int_{\ul y}^{0} \gamma (\gamma - 1) e^{(\gamma - 1) y} \tilde J(y) dy+\gamma  \bar F_{(1)}(x_B)^{-1} f_{(1)}(x_B) \bar J(x_B) 
\]
where $\tilde J(y) = \bar J (t(y))$, \[ \ul y < 0, \quad\quad \tilde J(\ul y) = 0 .\] Taking the derivative of the slope $s'(x_B)$ with respect to $\gamma$, and using the estimate
\begin{align*}
\frac{d}{d\gamma} \Big(  \gamma (1-\gamma) e^{(\gamma - 1)y} \Big) &= \Big( 2 \gamma - 1+\gamma(\gamma - 1)\Big) 
e^{(\gamma - 1) y}
\\&\geq \Big(1+(\gamma - 1)y \Big)e^{(\gamma - 1) y}
\end{align*}
we obtain the estimate
\begin{align*}
\frac{ds'(x_B)}{d \gamma} &\geq  f_{(1)}(x_B)\int_{\ul y}^{0} \Big(1+(\gamma - 1)y \Big)e^{(\gamma - 1) y}
\tilde J(y) dy+\bar F_{(1)}(x_B)^{-1} f_{(1)}(x_B) \bar J(x_B).
\end{align*}
The second term above is positive. As far as the first term, the extended mean-value theorem for integrals implies for some $a \in [\ul y,0]$\footnote{The second mean-value theorem for integrals states that $\int_a^b f(t)g(t)dt = g(a) \int_a^c f(t)dt+ g(b) \int_c^b f(t)dt$ whenever $f,g$ are continuous functions on $[a,b]$. See Theorem 2.12.17 on p.150 in \cite{bogachev2007measure}. Here, this theorem is applied with $g(t) = \tilde J(t)$, taking into account $\tilde J(\ul y) = 0$.}
\begin{align*}
\int_{\ul y}^{0} \Big(1+(\gamma - 1)y \Big)e^{(\gamma - 1) y}
\tilde J(y) dy &= \tilde J(0) \int_{\ul y}^{0} \Big(1+(\gamma - 1)y \Big)e^{(\gamma - 1) y} dy
\\&=J(x_B)  \int_{\ul y}^{0} d\Big(ye^{(\gamma - 1) y}\Big)
\\&=-J(x_B) a e^{(\gamma - 1) \ul a}\geq 0
\end{align*}
where the last line follows from $a<0$. So we conclude \[ \frac{ds'(x_B)}{d \gamma}> 0.\] Since $\gamma = 1/\alpha$, this implies $\frac{ds'(x_B)}{d \alpha}<0$, which in turn implies that the slope of the inverse $s^{-\mathbbm 1}(x_S)$ is increasing in $\alpha$. Since $p_S(x_S;\alpha) = s^{-\mathbbm 1}(x_S)+\alpha x_S$, we conclude that the slope of $p_S(x_S;\alpha)$ is increasing in $\alpha$. Since $p_S(0) =  \ul x_B$ is independent of $\alpha$, this implies $p_S(x_S;\alpha)$ increases in $\alpha$.

\end{proof}


\begin{proof}[Proof of Lemma 1] Indifference condition \eqref{eq:sindif0}  can be equivalently stated as
\begin{align}
  (\hat p(\underline p_S) - \underline x_S)(1-F_{(1)}(X_B^*(p_S^*(\underline x_S)))) = (\hat p(v_J) -
  \underline x_S)(1-F_{(1)}(\overline x_B)), \label{eq:sindif}
\end{align}
where $\hat p(p)$ denotes the equilibrium price received by the bank
\emph{conditional} on winning the auction with a reserve $p$. 
% As a matter of fact, we impose a stronger indifference condition for
% the bank: in our equilibrium, the bank with $\underline x_S$ will be
% indifferent over \emph{any} price in $[\underline p_S,\ol x_B]$.
The bank's indifference condition \eqref{eq:sindif} defines $\ol x_B$
as an implicit function of $\underline x_S$.
% and therefore, through $\underline x_S = p_S^{*-1}(\bar p_B^*(x_B^1))$, as a
% function of the lower cutoff $x_B^1$.
This function is denoted as $y_B(\cdot)$.  The indifference condition \eqref{eq:sindif}
   can be re-written as
\[
F_{(1)}(\ol x_B)=1-\frac{\Pi_S(\underline x_S)}{\hat p(v_J)-\underline x_S}
\]
where we denoted the broker's type that corresponds to $\underline x_S$ as $\tilde x_B(\underline x_S) =X_B(p_S^*(\underline x_S)))$, and
\[ \Pi_S(\underline x_S) = (\hat
p(p_S^*(\underline x_S))-x_S)(1-F_{(1)}(\tilde x_B(\underline x_S)) .\] 
Next, we show that
$\Pi_S(\underline x_S)/(\hat p(v_J)-\underline x_S)$ is increasing in $\underline x_S$, which
implies that $\ol x_B$ is decreasing in $\underline x_S$. The derivative of this
function is
\[
\frac{d}{d\underline x_S} \frac{\Pi_S(\underline x_S)}{\hat p(v_J)-\underline x_S} = 
\frac{\Pi_S'(\underline x_S)(\hat p(v_J)-\underline x_S)+\Pi_S(\underline x_S)}{(\hat p(v_J)-\underline x_S)^2}
\]

The envelope theorem implies $\Pi_S'(\underline x_S) =
-(1-F_{(1)}(\tilde x_B(\underline x_S)))$. So the numerator is equal to
\begin{align*}
&  \Pi_S(\underline x_S)-(\hat p(v_J)-\underline x_S)(1-F_{(1)}(\tilde x_B(\underline x_S))) \\
& = \Pi_S(\underline x_S)-(\hat p(p_1)-\underline x_S)(1-F_{(1)}(\tilde x_B(\underline x_S))) - (\hat p(v_J)-\hat p(p_1))(1-F_B(\tilde x_B(\underline x_S)))
\\
& = -(\hat p(v_J)-p_1)(1-F_{(1)}(\tilde x_B(\underline x_S))) < 0
\end{align*}
where the inequality follows since $p_1<v_J$ and $\hat p(\cdot)$ is a
an increasing function. Hence, $y_B(\underline x_S)$ is a \emph{decreasing} function.


The broker's indifference condition \eqref{eq:bindif0}, $H(\underline x_S,\ol x_B) = v_J$, defines
$\ol x_B$ as an implicit function of $\underline x_S$, \[ \ol x_B=z(\underline x_S).\] Indeed, we have $H(\ul x_S,v_J)<0$ and, for $\ol x_B \geq v_J$,  \begin{align*}
\frac{\partial H(\underline x_S,\ol x_B)}{\partial
    \ol x_B} &= \Big(  u_B(\ol x_B,\ol x_B) - \ol x_B \Big)f_S(\ol x_B)+\int_{\ul x_S}^{\ol x_B} \frac{\partial u_B(\ol x_B,x_S)}{\partial \ol x_B} f_S(x_S) dx_S
 \\
 &=   \int_{\ul x_S}^{\ol x_B} \frac{\partial u_B(\ol x_B,x_S)}{\partial \ol x_B} f_S(x_S) dx_S
 \\
 &\geq \alpha (F_S(\ol x_B) - F_S(v_J)),
\end{align*}
where we have used the assumption that $\partial u_B/\partial x_B \geq \alpha >0$. By integration, it then follows that for $\ol x_B \geq v_J$,
\[ 
H(\ul x_S, \ol x_B) \geq H( \ul x_S,v_J) + \int_{v_J}^{\ol x_B} (F_S(y) - F_S(v_J))dy \to \infty 
\]
as $\ol x_B \to \infty$. So for a given $\ul x_S < v_J$, $H(\ul x_S,\ol x_B)$ is an increasing function of $\ol x_B$, tending to $\infty$, with $H(\ul x_S,v_J) < 0$. This implies that the equation $H(\ul x_S, \ol x_B) = 0$ defines $\ol x_B$ as an implicit function of $\ul x_S$. This function will be denoted as $z(\ul x_S)$,
\[ H( \ul x_S, z_B(\ul x_S)) = 0 . \]

We now show that $z_B(\cdot)$ is an \emph{increasing} function. This will follow from the fact that
$ H(\underline x_S,\ol x_B)$ defined in \eqref{eq:bindif0}  is an
increasing function in first two arguments. We have already shown that $H(\ul x_S,\ol x_B)$ is increasing in $\ol x_B$ for $\ol x_B \geq v_J$. Now
\begin{align*}
  \frac{\partial H(\underline x_S,\ol x_B)}{\partial
    \ul x_S}= (v_J-u_B(\ol x_B,\ul x_S))f_S(\ul x_S).
    \end{align*}
We claim that $u_B(\ol x_B,\ul x_S)<v_J$, so that $H(\ul x_S, \ol x_B)$ is indeed increasing in $\ul x_S$. We argue by contradiction. If not, we would have 
\begin{align} 
H(\ul x_S, \ol x_B) \geq \int_{v_J}^{\ol x_B} \Big(  u_B(\ol x_B, x_S) - x_S \Big) f_S(x_S)dx_S,\label{eq:Hinproof}
\end{align}
Given our assumption $\partial u_B/\partial x_S <1$, we have for $x_S < \ol x_B$,
\begin{align*}
u_B(\ol x_B, x_S)& = u_B(\ol x_B, \ol x_B) - \int_{x_S}^{\ol x_B} \frac{\partial u_B(\ol x_B,x_S)}{\partial x_S} dx_S
\\
&> \ol x_B - \int{x_S}{\ol x_B} dx_S = x_S.
\end{align*}
Substituting this bound into \eqref{eq:Hinproof}, we get
\begin{align*}
H(\ul x_S, \ol x_B) &> \int_{v_J}^{\ol x_B} \Big( u_B(\ol x_B,x_S) - x_S \Big) f_S(x_S)dx_S
\\
&>0,
\end{align*}
a contradiction to $H(\ul x_S, \ol x_B) = 0$. So we conclude 
\[ \frac{\partial H(\underline x_S,\ol x_B)}{\partial
    \ul x_S}>0.
\]
The Implicit Function Theorem now implies that $z_B(\cdot)$ is increasing:
\[ z_B'(\ul x_S) = - \frac{\partial H/\partial \ul x_S}{\partial H /\partial \ol x_B}>0. \]    

\begin{figure}[ht]
\centering
\includegraphics[scale = 0.6]{graphics/intersection.pdf}
\caption{Functions $y_B(\cdot)$ and $z_B(\cdot)$.}
\label{fig:intersection}
\end{figure}


Thus $y_B(\ul x_S)$ and $z_B(\ul x_S)$ are both continuous, and are, respectively,  decreasing and increasing functions. Continuing with the proof, let $\hat x_S = X_S^*(v_J)$ and $\hat x_B = X_B^*(v_J)$ be the bank's and broker's types in the bank-seller equilibrium ($v_J = 0$). The lower cutoff values are restricted between
the lowest possible value $0$, and $\hat x_S$. Both mapping $y_B(\cdot)$ and $z_B(\cdot)$ are defined on the domain $[0,\hat x_S ]$. 



Since the $\hat x_B$ type breaks even if the bank drops out at $v_J$ in the bank-seller equilibrium ($v_J = 0$), we must have $H(\hat x_S,\hat x_B)>0$. The monotonicity of $H(\underline x_S,\ol x_B)$ in $\ol x_B$ implies
$z_B(\hat x_S)<\hat x_B$. At the same time, the definition of $y_B(\cdot)$ implies $\hat x_B = y_B(\hat x_S)$, and it follows that $z_B(\hat x_S)<y_B(\hat x_S)$. 
Refer to Figure~\ref{fig:intersection}. In view of the monotonicity of $y_B(\cdot)$ and $z_B(\cdot)$, there are two possibilities. First, if $z_B(0) \geq  y_B(0)$, then type-$0$ bank prefers to bid $\ul p$ rather than $v_J$. In this case, the curves $y_B(\ul x_S)$ and $z_B(\ul x_S)$ have a unique intersection given by 
\begin{align}
  \ol x_B = y_B(\underline x_S)=z_B(\underline x_S), \label{eq:intersection}
\end{align} 
with $\ol x_S \in (0,\hat x_S)$. If, on the other hand, $z_B(0) <  y_B(0)$, then type-$0$ bank prefers to bid $v_J$, so that the equilibrium involves $\ol x_B = 0$ and the bunching extends all the way to $x_S = 0$.
%
%The equilibrium cutoffs $\underline x_S$ and $\ol x_B$ are given by the
%intersection of the graphs of $y_B(\cdot)$ and $z_B(\cdot)$,


\end{proof}

%\section{Average Bidding Function in the Presence of Heterogeneity}\label{sec:avg-bid-heterogeneity}
%
%It may be reasonable to expect that even for non-securitized mortgages, some properties are better characterized by symmetric information. So as before, assume that a certain fraction of properties corresponds to symmetric information ($\alpha = 0$), while the remaining properties are better described by asymmetric information ($\alpha = 1$). The parameter $\alpha$ represents unobserved heterogeneity (to the econometrician). Then 
%\begin{align*}
%G_S^w(p_S(x_S;\alpha)|\alpha) = F_S(x_S|\alpha), \quad\quad \alpha \in \{ 0,1 \}
%\end{align*}
%Averaging over the values of $\alpha$, we have (keeping in mind that $p_S(x_S; 0) = p_S^0(x_S)$ and $p_S(x_S;1) = p_S(x_S)$ in our previous notation),
%\[ 
%G_S^w(p_S^0(x_S)|0)\mathbbm P(\alpha = 0) + G_S^w(p_S(x_S)|1)\mathbbm P(\alpha = 1) = F_S(x_S)
%\]
%Since the signalling premium implies $p_S^0(x_S) < p_S(x_S) $, it follows that there exists $\hat p_S(x_S)$, 
%\[ p_S^0(x_S) < \hat p_S(x_S) < p_S(x_S) \]
%such that
%\[  \hat p_S(x_S)  = G_S^{w-\mathbbm 1}(F_S(x_S)) \]
%and Hypothesis \ref{hyp:sp} continues to hold, except that the identifiable object $G_S^{w-\mathbbm 1}(F_S(x_S))$ is now interpreted as $\hat p_S(x_S)>p_S^0(x_S)$.
%%

\end{document}
